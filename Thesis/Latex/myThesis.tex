  \documentclass[12pt,a4paper,final]{article}
\usepackage[utf8]{inputenc}
%\usepackage[T1]{fontenc} % O%utput font encoding for international characters
\usepackage[ english, greek]{babel}
%\usepackage{mathpazo} % Palatin%o 
\usepackage{pdflscape}
\usepackage{amsmath}
\usepackage{amsfonts}
\usepackage{amssymb}
\usepackage{color}
\usepackage{fancyhdr}
\usepackage{longtable}
\fancyhf{}
\lhead{\foreignlanguage{greek}{Αναλυτική Επεξεργασία Δεδομένων Της Υπηρεσίας \textit{ΠΥΘΙΑ}}}
\rfoot{\foreignlanguage{greek}{Σελίδα \thepage}}
\lfoot{\foreignlanguage{greek}{Ψευτέλης Θεόδωρος}}
\pagestyle{fancy}
\usepackage{graphicx}
\usepackage{caption}
\usepackage{subcaption}
\graphicspath{{./figures/post/}}
\usepackage[backend=biber,style=alphabetic,sorting=ynt]{biblatex}
\usepackage{algorithm}
\usepackage{algorithmic}
\renewcommand{\algorithmicforall}{\textbf{for each}}
\addbibresource{myBib.bib}
\usepackage{listings}
\definecolor{codegreen}{rgb}{0,0.6,0}
\definecolor{codegray}{rgb}{0.5,0.5,0.5}
\definecolor{backcolour}{rgb}{0.95,0.95,0.92}
\usepackage{listings}
\lstset{
language=R,                     
  backgroundcolor=\color{backcolour},   
    commentstyle=\color{codegreen},
    keywordstyle=\color{black},
    numberstyle=\tiny\color{codegray},
    stringstyle=\color{red},
    basicstyle=\footnotesize,
    breakatwhitespace=false,         
    breaklines=true,                 
    captionpos=b,                    
    keepspaces=true,                 
    numbers=left,                    
    numbersep=5pt,                  
    showspaces=false,                
    showstringspaces=false,
    showtabs=false,                  
    tabsize=2
}
\begin{document}

%----------------------------------------------------------------------------------------
%	TITLE PAGE
%----------------------------------------------------------------------------------------

\begin{titlepage} % Suppresses displaying the page number on the title page and the subsequent page counts as page 1
	\newcommand{\HRule}{\rule{\linewidth}{0.5mm}} % Defines a new command for horizontal lines, change thickness here
	
	\center % Centre everything on the page
	
	%------------------------------------------------
	%	Headings
	%------------------------------------------------
	
	\textsc{\LARGE Αλεξάνδρειο Τεχνολογικό Εκπαιδευτικό Ίδρυμα Θεσσαλονίκης\\ Τμήμα Μηχανικών Πληροφορικής}\\[1.5cm] % Main heading such as the name of your university/college

	%------------------------------------------------
	%	Title
	%------------------------------------------------
	
	\HRule\\[0.4cm]
	
	{\huge\bfseries Αναλυτική Επεξεργασία Δεδομένων Της Υπηρεσίας \textit{ΠΥΘΙΑ}}\\[0.4cm] % Title of your document
	
	\HRule\\[1.5cm]
	
	%------------------------------------------------
	%	Author(s)
	%------------------------------------------------
	
	\begin{minipage}{0.4\textwidth}
		\begin{flushleft}
			\large
			\textit{Φοιτητής}\\
			\textsc{Ψευτέλης Θεόδωρος}\\ \bigskip% Your name
			\textit{Αριθμός Μητρώου}\\
			\textsc{113813} % Your name
		\end{flushleft}
	\end{minipage}
	~
	\begin{minipage}{0.4\textwidth}
		\begin{flushright}
			\large
			\textit{Επιβλέπων Καθηγητής}\\
			 \textsc{Δημήτριος Δέρβος}\\ \bigskip % Supervisor's name
			\textit{Συνεπιβλέπων}\\
			 \textsc{Σ. Ουγιάρογλου $PhD$ } % Supervisor's name
		\end{flushright}
	\end{minipage}
	
	% If you don't want a supervisor, uncomment the two lines below and comment the code above
	%{\large\textit{Author}}\\
	%John \textsc{Smith} % Your name
	
	%------------------------------------------------
	%	Date
	%------------------------------------------------
	
	\vfill\vfill\vfill % Position the date 3/4 down the remaining page
	
	{\large\today} % Date, change the \today to a set date if you want to be precise
	
	%------------------------------------------------
	%	Logo
	%------------------------------------------------
	
	\vfill\vfill
	\graphicspath{{./figures/}}
	\includegraphics[width=0.5\textwidth]{title}\\[1cm] % Include a department/university logo - this will require the graphicx package
	 
	%----------------------------------------------------------------------------------------
	
	\vfill % Push the date up 1/4 of the remaining page
	
\end{titlepage}

%----------------------------------------------------------------------------------------
\clearpage


\selectlanguage{english}
\begin{abstract}
At the Information Technology Department of the Thessaloniki ATEI the effort to improve the quality of the educational content and that of the teaching/training practice is intense and continuous. This is dictated by the rapid pace at which technology evolves and its impact on the job market requirements for I.T. skills and qualifications. 
The present thesis project focuses on the analytical processing of undergraduate and graduate exams data. The latter first undergo extensive preparation (cleansing and homogenization) for the results of the later data processing stages to be reliable and trustworthy. Next comes the statistical data processing stage. The first set of results obtained turn out to be particularly useful in sketching the students assessment 'profile' of each one course from one academic semester to the next and from one academic year to its next one. 
An objective set out as such from the beginning of the project has been to apply data mining algorithms in the processing of students examinations data. As it has turned out, the data preparation and the subsequent statistical processing stages have required more time and effort than what was originally expected. In this respect, the data mining stage has been restricted to association rules mining. Still, the rules discovered are informative and useful in the direction of assessing the quality of the department's undergraduate and graduate course curricula. In addition, the results obtained can be utilized in devising and information service dor the students to plan their study in a way that maximizes their benefit from enrolling in the degree program. In this respect, the present thesis project may be taken to comprise a first, pilot attempt to apply data mining techniques to students examinations data at a typical Higher Education Institution. 
\end{abstract}
\selectlanguage{greek}
\clearpage


\begin{abstract}
Στο ΑΤΕΙ/Θ, στο Τμήμα Μηχανικών Πληροφορικής Τ.Ε. ειδικότερα, 
καταβάλλεται συνεχής προσπάθεια για τη βελτίωση του παρεχόμενου 
εκπαιδευτικού έργου και του επιπέδου σπουδών και κατάρτισης των 
αποφοίτων του. Αυτό επιτάσσουν η διαρκής εξέλιξη της τεχνολογίας και ο 
αντίκτυπός της στις απαιτήσεις της σύγχρονης αγοράς εργασίας. 
Αντικείμενο της παρούσας πτυχιακής εργασίας συνιστά η επεξεργασία 
δεδομένων βαθμολογιών εξετάσεων προπτυχιακών μαθημάτων και 
μεταπτυχιακών μαθημάτων του Τμήματος.
Τα παραπάνω βαθμολογικά δεδομένα υπέστησαν πρώτα μία εκτεταμένη 
προεπεξεργασία καθαρισμού, διόρθωσης και ομογενοποίησης. Παράλληλα, το στάδιο της προεπεξεργασίας των 
δεδομένων περιελάμβανε και περιγραφικού τύπου στατιστική επεξεργασία 
από την οποία προέκυψαν εξαιρετικά χρήσιμα στοιχεία για τη βαθμολογική 
``συμπεριφορά" του κάθε ενός μαθήματος από την μία εξεταστική περίοδο 
στην επόμενή της και από το ένα ακαδημαϊκό έτος στο επόμενό του.
Απώτερο στόχο αποτελεί η επεξεργασία των βαθμολογικών δεδομένων με 
αλγόριθμους Εξόρυξης Πληροφορίας (\foreignlanguage{english}{Data Mining}). Προς την κατεύθυνση 
αυτή, χρησιμοποιήθηκε η τεχνική της παραγωγής συνδυαστικών κανόνων 
(\foreignlanguage{english}{association rules}) με τη χρήση του αλγόριθμου \foreignlanguage{english}{Apriori}. Στην εργασία 
γίνεται αναφορά για την πληροφορία η οποία εξήχθη και για τη 
χρησιμότητά της για τη βελτίωση του παρεχόμενου εκπαιδευτικού έργου 
και των υπηρεσιών πληροφόρησης των φοιτητών του Τμήματος. Με αυτήν την 
έννοια, η παρούσα εργασία συμβάλλει στην προτυποποίηση της χρήση 
τεχνικών εξόρυξης πληροφορίας από τα βαθμολογικά δεδομένα  ενός τριτοβάθμιου εκπαιδευτικού ιδρύματος.
\end{abstract}




\clearpage
\renewcommand{\abstractname}{Ευχαριστίες}
\begin{abstract}
Θα ήθελα να εκφράσω την βαθύτατη ευγνωμοσύνη μου στους επιβλέποντες καθηγητές μου, κύριο Δημήτριο Δέρβο και κύριο Στέφανο Ουγιάρογλου για τον προσωπικό χρόνο που αφιέρωσαν, τη συνεχή καθοδήγηση, την υπομονή  και την επιμονή τους. Η συμβολή τους ήταν καθοριστική για την ολοκλήρωση αυτής της πτυχιακής εργασίας. 
\end{abstract}
\clearpage
\tableofcontents
\newpage


\thispagestyle{empty}
 
\listoffigures
 
\listoftables
 
\newpage

\section{Εισαγωγή}
\paragraph{}
Τα εκπαιδευτικά ιδρύματα διαχειρίζονται δεδομένα τα οποία μπορούν να χρησιμοποιηθούν για την αξιολόγηση και τη βελτίωση της εκπαιδευτικής διαδικασίας. Πιο συγκεκριμένα, είναι οι  πληροφορίες των φοιτητών όπως η ηλικία, το φύλλο ή ακόμα εάν φοίτησαν σε γενικό ή επαγγελματικό λύκειο. Επίσης στοιχεία για την επίδοση των φοιτητών στα μαθήματα, τον βαθμό πτυχίου ακόμα και το πόσα χρόνια διήρκεσαν οι σπουδές τους. Όλα τα παραπάνω δεδομένα μπορούν να αποκτηθούν από το πληροφοριακό σύστημα (ΠΣ) του τμήματος.
\paragraph{}
Η γνώση που κρύβεται μέσα στα ακαδημαϊκά δεδομένα μπορεί να παίξει σημαντικό ρόλο στο σύνολο των ενεργειών που απαιτούνται για την αναβάθμιση της εκπαιδευτικής διαδικασίας των ιδρυμάτων\foreignlanguage{english}{\cite{students_learning_case6}}. 
Οι εκπαιδευτικοί, μέσω της ανάλυσης δεδομένων\foreignlanguage{english}{\cite{miningEducation1}}, έχουν την δυνατότητα να προσεγγίσουν τα γραπτά των φοιτητών από διαφορετικές πλευρές και κατά επέκταση να κατανοήσουν καλύτερα την \textit{βαθμολογική συμπεριφορά}βαθμολογική συμπεριφορά των φοιτητών ή ακόμα και να εντοπίσουν  πιθανές αιτίες που συνδράμουν στην διακύμανση της επίδοσης τους. Οι εκπαιδευτικοί λαμβάνοντάς υπόψιν την καινούργια πληροφορία θα είναι σε θέση να προσαρμόζουν την εκπαιδευτική διαδικασία πάνω στις, όλο και περισσότερο μεταβαλλόμενες, ανάγκες των σύγχρονων φοιτητών.

\paragraph{}
Τα εκπαιδευτικά ιδρύματα  για να βελτιώσουν την αποτελεσματικότητα και την συνολική εκπαιδευτική τους διαδικασία πρέπει πρώτα να αξιολογήσουν τις ήδη υπάρχουσες μεθόδους διδασκαλίας που χρησιμοποιούν\foreignlanguage{english}{ \cite{educational_mining5}}. Με αφορμή την εσωτερική αξιολόγηση του τμήματος και την ανάγκη για εύρεση νέων  τρόπων και τεχνικών για την εκμετάλλευση των δεδομένων προς όφελος των φοιτητών, μας δημιουργήθηκε η ανάγκη και το κίνητρο για την  πραγματοποίηση αυτής της εργασίας.
\paragraph{}
Σε αυτή την εργασία εφαρμόζουμε ένα συνδυασμό τεχνικών ανάλυσης με σκοπό να δούμε τα βαθμολογικά δεδομένα των φοιτητών από διαφορετικές οπτικές γωνίες. Αρχικά με περιγραφική στατιστική\foreignlanguage{english}{ \cite{descriptiveStats}}  για να εξάγουμε χρήσιμη πληροφορία σχετικά με την επίδοση των φοιτητών στα μαθήματα. Επίσης, σε αυτό το κομμάτι δημιουργήσαμε  γραφικές παραστάσεις  για την πληρέστερη κατανόηση των αποτελεσμάτων. Έπειτα συνεχίζουμε την ανάλυση με την εξαγωγή κανόνων συσχέτισης\foreignlanguage{english}{ \cite{associatio_rules7}} με σκοπό να βρούμε συσχετίσεις και πρότυπα σχετικά με μαθήματα που  οι φοιτητές τείνουν να πετυχαίνουν υψηλή επίδοση. Αξιοποιώντας τις   δυνατότητες της χρήσης τεχνικών εξόρυξης πληροφορίας στα βαθμολογικά δεδομένα του εκπαιδευτικού ιδρύματος, θεωρούμε ότι συμμετέχουμε ενεργά στη βελτίωση των εκπαιδευτικών διαδικασιών του τμήματος. 

\paragraph{}
Η εργασία οργανώθηκες ως εξής: Στην  ενότητα 2 παρουσιάζονται τα εργαλεία και οι τεχνικές που χρησιμοποιήθηκαν. Στο ενότητα 3 αναλύεται η περιγραφική στατιστική και παρουσιάζονται τα αποτελέσματά της. Στην ενότητα 4 παρουσιάζεται  ο αλγόριθμος  εξόρυξης κανόνων συσχέτισης  \foreignlanguage{english}{Apriori} και η εφαρμογή του στα ακαδημαϊκά δεδομένα. Τέλος στην ενότητα 5 βρίσκονται τα συμπεράσματα αυτής της εργασίας και προτάσεις για μελλοντικές εργασίες.

\clearpage

\section{Εργαλεία και Τεχνικές}
\subsection{Η γλώσσα Προγραμματισμού \foreignlanguage{english}{R} και το \foreignlanguage{english}{RStudio}}
 H $R$  \foreignlanguage{english}{\cite{R}} είναι ένα λογισμικό ανοιχτού κώδικα,  για την στατιστική και αναλυτική επερξεργασία δεδομένων. Περιλαμβάνει μια γλώσσα προγραμματισμού, συνδεσιμότητα με άλλες γλώσσες και δυνατότητα εντοπισμού σφαλμάτων. Η $R$ έχει πολλές ομοιότητες με την γλώσσα $S$ \foreignlanguage{english}{\cite{S}} και  θεωρείται ως μια διαφορετική υλοποίησή της.
  Βασικοί τύποι που παρέχει η $R$ για τον χειρισμό των δεδομένων είναι:
\begin{itemize}
	\item \foreignlanguage{english}{vectors}
	\item \foreignlanguage{english}{matrices}
	\item \foreignlanguage{english}{data frames}
	\item \foreignlanguage{english}{arrays}
	\item \foreignlanguage{english}{lists}
	\item \foreignlanguage{english}{factors}
\end{itemize}    
Ένα από τα βασικά πλεονεκτήματα της $R$ είναι η επεκτασιμότητά της. Συγκεκριμένα, μπορεί ο καθένας να προσθέσει λειτουργικότητα με την δημιουργία πακέτου ή  να χρησιμοποιήσει στον κώδικά του κάποιο από τα ήδη υπάρχοντα
(πχ. το \foreignlanguage{english}{ggplot2} για την παραγωγή γραφικών).\\
Στον παρακάτω κώδικα φαίνεται η ευκολία με την οποία μπορεί κάποιος να χρησιμοποιήσει ένα υπάρχον πακέτο:
\selectlanguage{english}
\begin{lstlisting}[language=R]
# Download package
install.packages(ggplot2)
# Use package
library(ggplot2)
\end{lstlisting}
\selectlanguage{greek}
Επίσης, ένα άλλο θετικό είναι ότι η $R$ παρέχει ενσωματωμένη συλλογή από εργαλεία για την αναλυτική επεξεργασία δεδομένων και την εξόρυξη χρήσιμης πληροφορίας. \medskip

Ένα από τα μειονεκτήματα της $R$ είναι ότι, καθώς είναι ανοιχτού κώδικα λογισμικό, δεν 
γίνεται κάποιος έλεγχος ποιότητας στα πακέτα, με συνέπεια να υπάρχουν πακέτα χαμηλής ποιότητας. Επίσης οι εντολές στην $R$ δεν δίνουν αρκετή σημασία στην διαχείριση της μνήμης, με αποτέλεσμα μερικές φορές να καταναλώνεται όλη η διαθέσιμη μνήμη του υπολογιστή. \medskip

Παράδειγμα κώδικα $R$ για την παραγωγή γραφικών (Σχήμα \ref{fig:simple_plot}):
\selectlanguage{english}	
\begin{lstlisting}[language=R]
# Get a random log-normal distribution
r <- rlnorm(1000)

# Get the distribution without plotting it using tighter breaks
h <- hist(r, plot=F, breaks=c(seq(0,max(r)+1, .1)))

# Plot the distribution using log scale on both axes, and use
# blue points
plot(h$counts, log="xy", pch=20, col="blue",
     main="Log-normal distribution",
     xlab="Value", ylab="Frequency")
\end{lstlisting}

\selectlanguage{greek}
\graphicspath{{./figures/}}
\begin{figure}[h]
         \centering
         \includegraphics[width=\textwidth]{simplePlot}
         \caption{Παράδειγμα Γραφήματος}
         \label{fig:simple_plot}
\end{figure}

\clearpage
Για την διεκπεραίωση της παρούσας εργασίας χρησιμοποιήθηκαν τα παρακάτω πακέτα:
\begin{itemize}
	\item $arules$ \cite{arules} Παρέχει λειτουργικότητα για την εξόρυξη κανόνων συσχέτισης. 
	\item $arulesViz$ \foreignlanguage{english}{\cite{arulesViz}} Παρέχει λειτουργικότητα για την παραγωγή γραφικών από τους κανόνες συσχέτισης. 
	\item $ggplot2$ \foreignlanguage{english}{\cite{ggplot2}}  Πακέτο για την παραγωγή γραφικών.
	\item $tidyverse$ \foreignlanguage{english}{\cite{tidyverse}} Είναι συλλογή πακέτων, ειδικά σχεδιασμένα για ανάλυση δεδομένων.
	\item $stringr$ \foreignlanguage{english}{\cite{stringr}} Πακέτο για τον χειρισμό συμβολοσειρών ($Strings$)
	\item $plyr$ \foreignlanguage{english}{\cite{plyr}} Παρέχει εργαλεία για τον διαχωρισμό,εφαρμογή και συνδυασμό δεδομένων.
	\item $extrafont$ \foreignlanguage{english}{\cite{extrafont}} Πακέτο για τον χειρισμό γραμματοσειρών.
	\item $xlsx$ \foreignlanguage{english}{\cite{xlsx}} Πακέτο για τον χειρισμό (λειτουργίες $read/write$) αρχείων $Excel$.
\end{itemize}  
%RStudio
\paragraph{}
Η χρησιμοποίηση της $R$ έγινε μέσω του $RStudio$\foreignlanguage{english}{\cite{Rstudio}}. Το $RStudio$ είναι ένα ανοιχτού κώδικα προγραμματιστικό περιβάλλον για την χρησιμοποίηση της $R$ (\foreignlanguage{english}{Integrated Development Enviroment-IDE}). Μέσω του γραφικού του περιβάλλοντος καθιστά εύκολο τον χειρισμό των αντικειμένων και των δομών αποθήκευσης δεδομένων της $R$. Επίσης παρέχει ενσωματωμένες δυνατότητες για τον χειρισμό μεγάλων έργων λογισμικού που αποτελούνται από πολλά αρχεία καθώς επιτρέπει την εύκολη πρόσβαση σε αυτά  μέσα από ένα ενοποιημένο γραφικό περιβάλλον. Ένα ακόμα πλεονέκτημα του $RStudio$ μπορεί να θεωρηθεί η αυτό-συμπλήρωση  κώδικα καθώς βοηθάει σημαντικά στην γρηγορότερη ανάπτυξη  λογισμικού. Επιπλέον, σημαντικό εργαλείο που παρέχει το $RStudio$ είναι ο γρήγορος εντοπισμός σφαλμάτων ($debugging$) στον κώδικα.
  


\clearpage
\subsection{Περιγραφική Στατιστική}
\paragraph{}
Η περιγραφική στατιστική\foreignlanguage{english}{\cite{descriptiveStats}} είναι ένα θεμελιώδης στάδιο στην ανάλυση δεδομένων. Συγκεκριμένα περιλαμβάνει συνοπτικά στατιστικά όπως μέσο όρο, ελάχιστη, μέγιστη τιμή κ.α. Επίσης, περιλαμβάνει μεθόδους οπτικοποίησης των δεδομένων όπως οι γραφικές παραστάσεις. Είναι σημαντική καθώς μέσω αυτής εξάγονται χρήσιμες πληροφορίες οι οποίες βοηθούν  στην βαθύτερη κατανόηση των δεδομένων. Στην εργασία αυτή μέσω της περιγραφικής στατιστικής προσπαθήσαμε να δούμε καλύτερα την επίδοση των φοιτητών στα μαθήματα. Έτσι θα μπορέσουμε να οργανώσουμε καλύτερα τα δεδομένα για το δεύτερο στάδιο της εργασίας: την Εξόρυξη Κανόνων Συσχέτισης.

\subsection{Εξόρυξη Δεδομένων}
\paragraph{}
Εξόρυξη δεδομένων (\foreignlanguage{english}{Data Mining \cite{datamining}}) ονομάζεται η διαδικασία εξαγωγής χρήσιμης πληροφορίας  ανάμεσα από μεγάλους όγκους δεδομένων. Με άλλα λόγια μπορούμε να πούμε ότι το \foreignlanguage{english}{Data Mining} είναι η εξόρυξη γνώσης από ακατέργαστα δεδομένα. Κάποιοι υποστηρίζουν ότι εξόρυξη γνώσης από βάσεις δεδομένων (\foreignlanguage{english}{KDD) \cite{OugKDD8}} και εξόρυξη γνώσης (\foreignlanguage{english}{Data Mining}) είναι ισοδύναμες έννοιες. Ωστόσο υπάρχει και η άποψη ότι η εξόρυξη γνώσης από δεδομένα (\foreignlanguage{english}{Data Mining}) αποτελεί μέρος της εξόρυξη γνώσης από βάσεις δεδομένων (\foreignlanguage{english}{KDD) \cite{FayyadKDD}}.\\
Τα βασικά στάδια της Εξόρυξη δεδομένων είναι: 

\begin{itemize}
	\item Πρώτο στάδιο είναι η συλλογή των δεδομένων από διάφορές πηγές. Αυτές μπορεί να είναι βάσεις δεδομένων, αρχεία δεδομένων ή  ακόμα και ο παγκόσμιος ιστός.
	\item Επόμενο βήμα είναι ο καθαρισμός και μετασχηματισμός των δεδομένων (αν χρειάζεται). Για παράδειγμα, κανονικοποίηση τιμών, διαγραφή λανθασμένων τιμών, αναπλήρωση τιμών που λείπουν κ.α. Αξίζει να σημειωθεί ότι ο καθαρισμός των δεδομένων είναι ένα από τα σημαντικότερα στάδια της ανάλυσης  καθώς παίζει καθοριστικό ρόλο στην ποιότητα των αποτελεσμάτων. Επίσης περιλαμβάνει τη δημιουργία νέων μεταβλητών, εάν αυτό κρίνεται απαραίτητο. Αυτή η διαδικασία μπορεί να καταλαμβάνει μέχρι και το 80\% του συνολικού χρόνου της ανάλυσης. 
	\item Στη συνέχεια ακολουθεί η μοντελοποίση, η επιλογή και η εφαρμογή των κατάλληλων τεχνικών εξόρυξης. Βασικές τεχνικές εξόρυξης πληροφορίας από δεδομένα είναι οι εξής:
	\begin{itemize}
		\item Κανόνες Συσχέτισης \foreignlanguage{english}{\cite{associationRules_apriori4},\cite{associatio_rules7}}
		\item Συσταδοποίηση \foreignlanguage{english}{\cite{clustering}}
		\item Κατηγοριοποίηση \foreignlanguage{english}{\cite{classification}}
		\item Παλινδρόμηση \foreignlanguage{english}{\cite{regressionModel}}
	\end{itemize}	
	\item Τελευταίο στάδιο είναι  η ερμηνεία και η αξιολόγηση των αποτελεσμάτων. Η εξόρυξη πληροφορίας βρίσκει εφαρμογή  σε  διάφορους τομείς όπως η εκπαίδευση \foreignlanguage{english}{\cite{educational_mining5}, \cite{miningEducation1}, \cite{OugKDD8}} η οικονομία \foreignlanguage{english}{\cite{economy}}, οι επιχειρήσεις λιανικής πώλησης \foreignlanguage{english}{\cite{retail}} και ο αθλητισμός \foreignlanguage{english}{\cite{sports}}.
	
\end{itemize}

\subsection{Εξόρυξη Κανόνων Συσχέτισης}
\paragraph{}
Η εξόρυξη κανόνων συσχέτισης είναι μια από τις πιο γνωστές και καλά μελετημένες τεχνικές εξόρυξης γνώσης από δεδομένα\foreignlanguage{english}{\cite{assoc_agrawal}}. Σκοπό έχει να ανακαλύψει κανόνες ανάμεσα στα δεδομένα που ικανοποιούν κάποια μεγέθη σημαντικότητας (\foreignlanguage{english}{support, confidence, lift}). Οι κανόνες συσχέτισης είναι κατάλληλοι για την εξόρυξη γνώσης από μη αριθμητικά δεδομένα.\\
Οι παραγόμενοι κανόνες έχουν την μορφή Εάν/Τότε (\foreignlanguage{english}{If/then})και αποτελούνται από το σώμα (Εάν) και την κεφαλή (Τότε) μέρος.\\
Για παράδειγμα:
\begin{itemize}
\item \textbf{Εάν}  ένας πελάτης αγοράζει ψωμί \textbf{Τότε} είναι πιθανότερο να αγοράζει και γάλα.  
\end{itemize}

 Οι εξόρυξη κανόνων συσχέτισης είναι επίσης γνωστή και ως \foreignlanguage{english}{\textit{Market Basket Analysis}} καθώς αρχικά εφαρμόσθηκε στο χώρο των επιχειρήσεων λιανικής\foreignlanguage{english}{\cite{apriori_basket2}} έχοντας ως στόχο την εύρεση προτύπων σχετικά με τον εντοπισμό προϊόντων όπου η αγορά του ενός προκαλεί την αγορά του άλλου. Υπάρχουν διάφοροι αλγόριθμοι για την εξόρυξη κανόνων συσχέτισης, μεταξύ των πιο γνωστών είναι ο  \foreignlanguage{english}{Apriori} τον οποίο και  χρησιμοποιούμε στην παρούσα εργασία. Σκοπός μας είναι να ανακαλύψουμε κανόνες της μορφής:
 \begin{itemize}
\item \textbf{Εάν} ένας φοιτητής έχει \textit{πολύ καλή} επίδοση στο μάθημα $X$, \textbf{Τότε} τείνει να έχει \textit{άριστη} επίδοση στο μάθημα $Y$. 
 \end{itemize}


\subsection{Πληροφορίες για τα Δεδομένων}
\paragraph{}
 Τα δεδομένα που χρησιμοποιήθηκαν  για την παρούσα εργασία  αποκτήθηκαν από το πληροφοριακό σύστημα του Αλεξάνδρειου Τεχνολογικού Εκπαιδευτικού Ιδρύματος Θεσσαλονίκης το οποίο ονομάζεται "Πυθία". Συγκεκριμένα περιλαμβάνει τις βαθμολογίες των προπτυχιακών φοιτητών του τμήματος Μηχανικών Πληροφορικής σε όλες τις εξεταστικές περιόδους από το χειμερινό  εξάμηνο του 2003 έως και το χειμερινό του 2017. Επιπλέον σε αυτά τα δεδομένα υπάρχουν πληροφορίες για τους φοιτητές  καθώς επίσης και πληροφορίες  για τα μαθήματα. Σε αυτό το χρονικό διάστημα που καλύπτουν τα δεδομένα έχουν εξεταστεί 3210 διαφορετικοί φοιτητές.

\paragraph{}
Τα δεδομένα για το μεταπτυχιακό πρόγραμμα σπουδών περιλαμβάνουν τα γραπτά των φοιτητών από το χειμερινό εξάμηνο του ακαδημαϊκού έτους 2013-2014 έως και το εαρινό εξάμηνο του 2016-2017 ακαδημαϊκού έτους. Το μεταπτυχιακό πρόγραμμα σπουδών περιλαμβάνει δέκα μαθήματα. Κατά τη χρονική περίοδο των τεσσάρων αυτών ακαδημαϊκών ετών έλαβαν μέρος σε εξετάσεις 83 φοιτητές.  


\section{Περιγραφική Στατιστική}
\subsection{Σύνολο Δεδομένων}
\paragraph{}
Τα δεδομένα που χρησιμοποιήθηκαν αφορούν τα ακαδημαϊκά έτη 2015-2016 και 2016-2017 στα οποία εφαρμόζεται το Π5 πρόγραμμα σπουδών. Το Π5 αποτελείται από 45 θεωρίες και 25 εργαστήρια. Κατά τη χρονική περίοδο αυτή πραγματοποιήθηκαν 8 εξεταστικές(4 ανά ακαδημαϊκό έτος). Συγκεκριμένα, μία εξεταστική για το εαρινό εξάμηνο, μία για το χειμερινό, μία εμβόλιμη στο χειμερινό εξάμηνο και τέλος η εξεταστική του Σεπτεμβρίου. Σε αυτό το διάστημα έλαβαν μέρος 1090 φοιτητές και βαθμολογήθηκαν 24615 γραπτά. Από το σύνολο της πληροφορίας που είχαμε διαθέσιμο κρατήσαμε τα εξής:
\begin{itemize}
\item \foreignlanguage{english}{ID} Φοιτητή
\item Τίτλος Μαθήματος
\item Εξεταστική Περίοδος 
\item Ακαδημαϊκό Έτος
\item Βαθμός (που βαθμολογήθηκε το γραπτό)
\end{itemize}

\subsection{Αθροιστικά Στατιστικά}
\paragraph{}
Αρχικά βγάλαμε αθροιστικά στατιστικά για κάθε μάθημα ξεχωριστά ανά ακαδημαϊκό έτος και ανά εξεταστική περίοδο. Πιο αναλυτικά, βρήκαμε το συνολικό αριθμό γραπτών, επίσης υπολογίσαμε το ποσοστό από τα γραπτά τα οποία βαθμολογήθηκαν στο διάστημα [0,1]. Θεωρήσαμε ότι τα γραπτά αυτά αφορούν φοιτητές οι οποίοι δεν αφιέρωσαν καθόλου χρόνο για να προετοιμαστούν για την εξέταση του συγκεκριμένου μαθήματος. Ως εκ τούτου θα επηρέαζαν αρνητικά την ποιότητα των αποτελεσμάτων μας. Έπειτα υπολογίσαμε το ποσοστό των γραπτών που βαθμολογήθηκαν με  βαθμό μεγαλύτερο ή ίσο του 5, και τέλος τον μέσο βαθμό των γραπτών.Στις δύο τελευταίες μετρήσεις δεν λήφθηκαν υπόψιν τα γραπτά που βαθμολογήθηκαν στο διάστημα [0,1]. Όλα τα παραπάνω αποτελέσματα τα συμπεριλάβαμε σε γραφήματα πυκνότητας (πχ. Σχήμα \ref{fig:exam_2016}) τα οποία δείχνουν τους βαθμούς με τους οποίους έχει περισσότερες πιθανότητες να βαθμολογηθεί ένας φοιτητής στο κάθε μάθημα με βάση τα δεδομένα.
\paragraph{}
Ένας άλλος τρόπος για να προσεγγίσουμε ποσοτικά τον βαθμό δυσκολίας των μαθημάτων, ήταν για κάθε μάθημα ξεχωριστά να υπολογίσουμε το ποσοστό των φοιτητών που πέρασαν και το ποσοστό αυτών που δεν πέρασαν  το μάθημα σε σχέση με τον αριθμό των εξετάσεων που έλαβαν μέρος. Αναλυτικότερα, να βρούμε τι ποσοστό από τους φοιτητές που έχουν εξεταστεί στο συγκεκριμένο μάθημα, έχει εξεταστεί μόνο μία φορά και το πέρασε, τι ποσοστό έχει εξεταστεί μια φορά αλλά δεν το έχει περάσει ακόμα, τι ποσοστό το πέρασε με την δεύτερη φορά , τι ποσοστό έχει εξεταστεί δύο φορές αλλά δεν το έχει περάσει ακόμα κ.ο.κ. \\
Για να υπολογίσουμε τα παραπάνω ποσοστά ορίσαμε ως αφετηρία το ακαδημαϊκό έτος 2015-2016.
\paragraph{}
Αρχικά ομαδοποιήσαμε τα δεδομένα ανά μάθημα, στη συνέχεια για κάθε μάθημα ξεχωριστά κρατήσαμε τους φοιτητές που έχουν εξεταστεί  τουλάχιστον μία φορά στο συγκεκριμένο μάθημα. Έπειτα για κάθε ένα από τους παραπάνω φοιτητές πήραμε τους βαθμούς  από όλα τα γραπτά του φοιτητή στο συγκεκριμένο μάθημα. Επόμενο βήμα ήταν να ελέγξουμε αν ένας φοιτητής έχει εξεταστεί πρώτη φορά στο μάθημα το  2015-2016. Όταν η παραπάνω υπόθεση ήταν αληθής ελέγξαμε αν υπάρχει βαθμολογία μεγαλύτερη ή ίση του πέντε, καθώς αυτό σημαίνει ότι έχει περάσει το μάθημα, αν υπήρχε το προσθέταμε στο ποσοστό των φοιτητών που έχουν περάσει το μάθημα με αριθμό προσπαθειών όσες είναι και ο συνολικός αριθμός των γραπτών του φοιτητής στο συγκεκριμένο μάθημα. Την ίδια διαδικασία ακολουθήσαμε  και για φοιτητές που έδωσαν το μάθημα  πρώτη φορά το 2016-2017. Τέλος τα παραπάνω  αποτελέσματα τα οπτικοποιήσαμε  με τη βοήθεια Γραφημάτων Ράβδων (\foreignlanguage{english}{barplots})(π.χ το Σχήμα
 \ref{fig:postBarpl}).
\clearpage

\subsection{Αποτελέσματα}
\subsubsection{Προπτυχιακό Πρόγραμμα Σπουδών}

Στο Σχήμα  \ref{fig:exam_2015}, το όνομα του μαθήματος και το ακαδημαϊκό έτος  αναγράφονται στην πρώτη γραμμή στο  επάνω αριστερό άκρο της εικόνας: \textit{Συστήματα Διαχείρισης Βάσεων Δεδομένων,2015-2016}. Στη δεύτερη γραμμή της ίδιας  λεζάντας  αναφέρεται ότι κατά την πρώτη εξεταστική περίοδο του εαρινού εξαμήνου προσήλθαν για να εξεταστούν  δεκατρείς (13) από τους εγγεγραμμένους στο μάθημα φοιτητές. Κανείς από τους δεκατρείς φοιτητές δεν βαθμολογήθηκε στο διάστημα[0,1]. Το ποσοστό των φοιτητών που βαθμολογήθηκαν στο διάστημα[0,1] παραλείπεται κατά τον υπολογισμό των δύο επόμενων δύο στατιστικών τιμών: 	του ποσοστού των φοιτητών οι οποίοι βαθμολογήθηκαν με βαθμό μεγαλύτερο ή ίσο της προβιβάσιμης τιμής πέντε (5.0, 30.\%) και τον μέσο βαθμό στην εν λόγω εξέταση[τρία κόμμα δύο, 3.2].\\
Αντίστοιχα για την πρώτη (Α) εξεταστική του χειμερινού εξαμήνου του μαθήματος συμμετείχαν στις εξετάσεις  εκατό εβδομήντα εφτά (177) φοιτητές/τριες, ένα ποσοστό (24.3\%) βαθμολογήθηκε στο διάστημα[0,1]. Το ποσοστό επιτυχίας στην εξέταση ήταν (44.5\%) και η μέση τιμή του βαθμού ήταν \textit{3.6}.\\
Για την την δεύτερη (Β) εξεταστική του εαρινού εξαμήνου του μαθήματος συμμετείχαν στις εξετάσεις  δύο φοιτητές/τριες, και οι δύο βαθμολογήθηκαν με βαθμό μεγαλύτερο ή ίσο του πέντε (5.0) και ο μέσος βαθμός τον γραπτών τους ήταν \textit{5.5}.\\
Για την την δεύτερη (Β εξεταστική) του χειμερινού εξαμήνου (εμβόλιμη) του μαθήματος συμμετείχαν στις εξετάσεις σαράντα πέντε (45) φοιτητές/τριες. Από αυτούς ένα ποσοστό (\textit{31.2 \%}) βαθμολογήθηκαν στο διάστημα[0, 1]. Το ποσοστό επιτυχίας στην εξέταση ήταν \textit{15.2\%} και η μέση τιμή του βαθμού ήταν \textit{2.2}. \\
Ανάλογα, στο Σχήμα \ref{fig:exam_2016} για το μάθημα \textit{Συστήματα Διαχείρισης Βάσεων Δεδομένων} κατά το ακαδημαϊκό έτος 2016-2017.\\
Για την πρώτη εξεταστική του χειμερινού εξαμήνου (Α ΧΕΙΜ) προσήλθαν  για να εξεταστούν 153 φοιτητές/τριες. Από αυτούς ένα ποσοστό \textit{2\%} βαθμολογήθηκε στο διάστημα[0,1]. Το ποσοστό επιτυχίας ήταν \textit{46\%} και η μέση τιμή του βαθμού ήταν \textit{4.7}.\\
Για την δεύτερη  εξεταστική του χειμερινού εξαμήνου (Β ΧΕΙΜ) προσήλθαν για να εξεταστούν 76 φοιτητές/τριες. Από αυτούς ένα ποσοστό \textit{5.3\%} βαθμολογήθηκε στο διάστημα[0,1].Το ποσοστό επιτυχίας ήταν \textit{52.8\%} και η μέση τιμή του βαθμού ήταν \textit{4.2}.\\
Για την πρώτη εξεταστική του εαρινού εξαμήνου (Α ΕΑΡ) προσήλθε για να εξεταστεί 1 φοιτητής/τρια ο οποίος βαθμολογήθηκε με \textit{6.2}.\bigskip

Παρακάτω ο κώδικας $R$ που παράγει τα  γραφήματα κατανομής συχνότητας του Σχήματος \ref{fig:exams_propt}.
\selectlanguage{english}
\begin{lstlisting}[language=R]
ggplot(course_df, aes(x=grade, colour=exam))+labs(y="%")+
    ggtitle(title)+
    stat_density(geom="line", position="identity")+
    scale_y_continuous(labels=scales::percent,limits = c(0,.5),
    breaks = seq(0,0.5,0.05))+
    scale_x_continuous("grade",limits = c(0,10),breaks = seq(0,10,1))+
    theme(plot.title = element_text(size=10),
          legend.position = "bottom")
\end{lstlisting}
\selectlanguage{greek}

\graphicspath{{./figures/post/}}
% per exam Density plots
\begin{figure}[h]
	\centering
	\begin{subfigure}[b]{0.475\textwidth}
		\centering
		\includegraphics[width=\textwidth]{exam_2015}
		\caption{Ακαδημαικό έτος 2015 - 2016}
		\label{fig:exam_2015}
	\end{subfigure}
	\hfill
	\begin{subfigure}[b]{0.475\textwidth}
		\centering
		\includegraphics[width=\textwidth]{exam_2016}
		\caption{Ακαδημαικό έτος 2016 - 2017}
		\label{fig:exam_2016}
	\end{subfigure}
	\caption{Προπτυχιακό Μάθημα: Κατανομή συχνότητας βαθμών ανά εξεταστική περίοδο}
	\label{fig:exams_propt}
\end{figure}
\clearpage


% Ανά  ακαδημαικό έτος 2015-16
Στο Σχήμα \ref{fig:year_2015} φαίνεται η συνάθροιση  των ποσοστών επιτυχίας και αποτυχίας στο μάθημα \textit{Συστήματα Διαχείρισης Βάσεων Δεδομένων} για το ακαδημαϊκό έτος  2015-2016.
Στο σύνολο των εξεταστικών περιόδων του ακαδημαϊκού έτους, βαθμολογήθηκαν διακόσια (240) γραπτά. Το \textit{24.2\%} των τελευταίων βαθμολογήθηκαν με βαθμό στο διάστημα[0, 1] και δεν λογίζονται στη συνέχεια. Πρόκειται για φοιτητές/τριες “επισκέπτες” (ίσως σε μεγάλο τυπικό εξάμηνο σπουδών) η αξιολόγηση των οποίων θεωρείται ότι συνιστά ένα είδος "θορύβου" στην αξιολόγηση που αποσκοπεί να αναδείξει το βαθμό δυσκολίας του μαθήματος. Το \textit{39\%}των υπόλοιπων γραπτών βαθμολογήθηκαν με προβιβάσιμο βαθμό και ο μέσος βαθμός με τον οποίο βαθμολογήθηκε γραπτό του συγκεκριμένου μαθήματος το συγκεκριμένο ακαδημαϊκό έτος \textit{3.4}\\
% Ανά ακαδημαικό έτος 2016-17
Αντίστοιχα στο Σχήμα \ref{fig:year_2016} φαίνεται η συνάθροιση των ποσοστών επιτυχίας και  αποτυχίας στο μάθημα \textit{Συστήματα Διαχείρισης Βάσεων Δεδομένων} για το ακαδημαϊκό έτος  2016-2017.
Στο σύνολο των εξεταστικών περιόδων του ακαδημαϊκού έτους, βαθμολογήθηκαν διακόσια (230) γραπτά. Το \textit{3\%} των τελευταίων βαθμολογήθηκαν με βαθμό στο διάστημα[0, 1] και δεν λογίζονται στη συνέχεια. Το \textit{48.4\%}των υπόλοιπων γραπτών βαθμολογήθηκαν με προβιβάσιμο βαθμό και ο μέσος βαθμός με τον οποίο βαθμολογήθηκε γραπτό του συγκεκριμένου μαθήματος το συγκεκριμένο ακαδημαϊκό έτος \textit{4.5}.\medskip

Παρακάτω ο κώδικας $R$ που παράγει τα  γραφήματα κατανομής συχνότητας ανά ακαδημαϊκό του Σχήματος \ref{fig:years}.

\selectlanguage{english}
% Per year density plots
\selectlanguage{english}
\begin{lstlisting}[language=R]
ggplot(course_df, aes(x=grade,color= "red"))+labs(y="%")+
	ggtitle(title)+
    stat_density(geom="line", position="identity",adjust = 0.5)+
    scale_y_continuous(labels=scales::percent,limits = c(0,.5),
    breaks = seq(0,0.5,0.05))+
    scale_x_continuous("grade",limits = c(0,10),breaks = seq(0,10,1))+
    theme(plot.title = element_text(size=10),
  	legend.position = "bottom")
\end{lstlisting}
\selectlanguage{greek}
\begin{figure}[h]
	\centering
	\begin{subfigure}[b]{0.475\textwidth}
		\centering
		\includegraphics[width=\textwidth]{year_2015}
		\caption{Ακαδημαικό έτος 2015 - 2016}
		\label{fig:year_2015}
	\end{subfigure}
	\hfill
	\begin{subfigure}[b]{0.475\textwidth}
		\centering
		\includegraphics[width=\textwidth]{year_2016}
		\caption{Ακαδημαικό έτος 2016 - 2017}
		\label{fig:year_2016}
	\end{subfigure}
	\caption{Προπτυχιακό Μάθημα: Κατανομή συχνότητας βαθμών ανά ακαδημαϊκό έτος}
	\label{fig:years}
\end{figure}

\clearpage

% Προπτυχιακό barplot 2015-2106
Το Σχήμα \ref{fig:barplot_2015} αναφέρεται στο (θεωρητικό) μάθημα \textit{Συστήματα Διαχείρισης Βάσεων Δεδομένων}. Αφορά φοιτητές/τριες που ξεκίνησαν να εξετάζονται στο εν λόγω μάθημα το ακαδημαϊκό έτος 2015-16. Το σύνολο των εν λόγω φοιτητών  συμπεριλαμβάνει εξίσου νέους (πρωτοεξεταζόμενους) και παλιότερους φοιτητές. Ξεκινά να τους "παρακολουθεί" από την αρχή του ακαδημαϊκού έτους 2015-2016 έως και την Β' εξεταστική περίοδο του ακαδημαϊκού έτους 2016-17. Τα ποσοστά που αναγράφονται στην πρώτη στήλη του ιστογράμματος διαβάζονται ως εξής: από τους φοιτητές που εξετάστηκαν  το ακαδημαϊκό έτος 2015-16 στο εν λόγω μάθημα, ένα ποσοστό 32\% το πέρασαν (με την "πρώτη") και ένα ποσοστό    23.3\% από αυτούς που εξετάστηκαν μία μόνον φορά στο μάθημα και δεν το έχουν περάσει ακόμη (δηλ. έως και τη Β' εξεταστική περίοδο  του ακαδημαϊκού έτους 2016-17).\\
Η δεύτερη στήλη του ιστογράμματος πληροφορεί ότι: από τους φοιτητές οι οποίοι εξετάστηκαν το ακαδημαϊκό έτος  2015-16 στο εν λόγω μάθημα, ένα ποσοστό 13.5\%  το πέρασαν (με τη \textit{δεύτερη}) και ένα ποσοστό 12.4\%  από αυτούς εξετάστηκαν δύο φορές στο μάθημα και  δεν το έχουν περάσει ακόμη (έως και τη Β' εξεταστική περίοδο του ακαδημαϊκού έτους 2016-2017).\\
Η τρίτη στήλη του ιστογράμματος πληροφορεί ότι: από τους φοιτητές οι οποίοι εξετάστηκαν το ακαδημαϊκό έτος  2015-16 στο εν λόγω μάθημα, ένα ποσοστό 9.3\%  το πέρασαν (με τη \textit{τρίτη}) φορά και ένα ποσοστό 4.1\%  από αυτούς εξετάστηκαν τρεις  φορές στο μάθημα και  δεν το έχουν περάσει ακόμη (έως και τη Β' εξεταστική περίοδο του ακαδημαϊκού έτους 2016-2017).\\
Η τέταρτη στήλη του ιστογράμματος πληροφορεί ότι: από τους φοιτητές οι οποίοι εξετάστηκαν το ακαδημαϊκό έτος  2015-16 στο εν λόγω μάθημα, ένα ποσοστό 2.6\%  το πέρασαν (με τη \textit{τέταρτη}) φορά και ένα ποσοστό 1.6\%  από αυτούς εξετάστηκαν τέσσερις  φορές στο μάθημα και  δεν το έχουν περάσει ακόμη(έως και τη Β' εξεταστική περίοδο του ακαδημαϊκού έτους 2016-2017).\\
Η πέμπτη στήλη του ιστογράμματος μας  πληροφορεί ότι: από τους φοιτητές οι οποίοι εξετάστηκαν το ακαδημαϊκό έτος  2015-16 στο εν λόγω μάθημα, το \textit{πέρασαν} όλοι όσοι εξετάστηκαν για πέμπτη φορά. \medskip

% Προπτυχιακό barplot 2016-2017
Αντίστοιχα, το Σχήμα \ref{fig:barplot_2016} αναφέρεται στο (θεωρητικό) μάθημα \textit{Συστήματα Διαχείρισης Βάσεων Δεδομένων}. Αφορά φοιτητές/τριες που ξεκίνησαν να εξετάζονται στο εν λόγω μάθημα το ακαδημαϊκό έτος 2016-17. Το σύνολο των εν λόγω φοιτητών  συμπεριλαμβάνει εξίσου νέους(πρωτοεξεταζόμενους) και παλιότερους φοιτητές. Ξεκινά να τους \textit{παρακολουθεί} από την αρχή του ακαδημαϊκού έτους 2016-17 έως και την Β' εξεταστική περίοδο του ίδιου ακαδημαϊκού έτους 2016-17. Τα ποσοστά που αναγράφονται στην πρώτη στήλη του ιστογράμματος διαβάζονται ως εξής:από τους φοιτητές που εξετάστηκαν  το ακαδημαϊκό έτος 2016-17 στο εν λόγω μάθημα, ένα ποσοστό 44\% το πέρασαν (με την \textit{πρώτη}) και ένα ποσοστό   29.3\% από αυτούς που εξετάστηκαν μία μόνον φορά στο μάθημα και δεν το έχουν περάσει ακόμη(δηλ. έως και τη Β' εξεταστική περίοδο  του ακαδημαϊκού έτους 2016-17).\\
Η δεύτερη στήλη του ιστογράμματος πληροφορεί ότι: από τους φοιτητές οι οποίοι εξετάστηκαν το ακαδημαϊκό έτος  2016-17 στο εν λόγω μάθημα, ένα ποσοστό 12.9\%  το πέρασαν (με τη \textit{δεύτερη}) και ένα ποσοστό 13.8\%  από αυτούς, εξετάστηκαν δύο φορές στο μάθημα και δεν το έχουν περάσει ακόμη(έως και τη Β' εξεταστική περίοδο του ακαδημαϊκού έτους 2016-2017).\medskip

Παρακάτω ο κώδικας $R$ που παράγει τα  γραφήματα του Σχήματος \ref{fig:postBarpl}.

\selectlanguage{english}
\begin{lstlisting}[language=R]
barpl <-ggplot()+geom_bar(data=Editdf,aes(x=try,fill=status,
	y=percent), stat = "identity")
barpl <- barpl + geom_text(data=Editdf ,aes(x = try, y = pos,
         label = paste0(percent,"%")), size=3,color="white")
  barpl <- barpl +theme(legend.position="bottom", 
  legend.direction="horizontal", legend.title = element_blank())
                        
  barpl <- barpl + labs(x = "Try", y = "Percentage") +
        scale_y_continuous(labels = dollar_format(suffix = "%",
    prefix = "")) +
    ggtitle(paste0(stud_passed2015[i,1],"(2015-16)"))+
    scale_fill_manual(values=fill) +
	theme(plot.title = element_text(size=11,face="bold"))+
    theme(axis.line = element_line(size=1, colour = "black"),
    panel.grid.major = element_blank(), panel.grid.minor =
    element_blank(),
    panel.border = element_blank(), panel.background = element_blank()) 
\end{lstlisting}
\selectlanguage{greek}

\begin{figure}[h]
	\centering
	\begin{subfigure}[b]{0.475\textwidth}
		\centering
		\includegraphics[width=\textwidth]{barplot_2015}
		\caption{Ποσοστά φοιτητών που εξετάστηκαν πρώτη  φορά το ακαδημαϊκό έτος  2015 - 2016}
		\label{fig:barplot_2015}
	\end{subfigure}
	\hfill
	\begin{subfigure}[b]{0.475\textwidth}
		\centering
		\includegraphics[width=\textwidth]{barplot_2016}
		\caption{Ποσοστά φοιτητών που εξετάστηκαν πρώτη φορά το ακαδημαϊκό έτος  2016 - 2017}
		\label{fig:barplot_2016}
	\end{subfigure}
	\caption{Προπτυχιακό Μάθημα: Προσπάθειες (άξονας \foreignlanguage{english}{Try}) και ποσοστά επιτυχίας/αποτυχίας εξετασθέντων (άξονας \foreignlanguage{english}{Percentage})}
	\label{fig:postBarpl}
\end{figure}
\clearpage

\subsubsection{Μεταπτυχιακό Πρόγραμμα Σπουδών}
% Μεταπτυχιακό density plots 
Στην υποενότητα αυτή παρουσιάζονται  τα αποτελέσματα για το Μεταπτυχιακό Πρόγραμμα Σπουδών από το ακαδημαϊκό έτος 2013 - 2014 έως και το 2016 - 2017 για το μάθημα \textit{Ανάκτηση Πληροφοριών στο Διαδίκτυο}\\

Στο Σχήμα \ref{fig:mscexam2013} απεικονίζονται τα στατιστικά  για το ακαδημαϊκό έτος 2013-2014. Στη λεζάντα  αναφέρεται ότι: Για την πρώτη εξεταστική του εαρινού εξαμήνου(Α ΕΑΡ) προσήλθαν  για να εξεταστούν 29 φοιτητές/τριες.
Από αυτούς ένα ποσοστό \textit{6.9\%} βαθμολογήθηκε στο διάστημα[0, 1].
Το ποσοστό επιτυχίας ήταν \textit{74.1\%} και η μέση τιμή του βαθμού ήταν \textit{5.3}.\\
Για την δεύτερη  εξεταστική του εαρινού εξαμήνου(Β ΕΑΡ) προσήλθαν για να εξεταστούν 10
φοιτητές/τριες. Από αυτούς τους φοιτητές κανείς δεν βαθμολογήθηκε στο διάστημα[0, 1].Το ποσοστό επιτυχίας ήταν \textit{80\%} και η μέση τιμή του βαθμού ήταν \textit{5.9}.\medskip

Στο Σχήμα \ref{fig:mscexam2014} απεικονίζονται τα στατιστικά  για το ακαδημαϊκό έτος 2014-2015. Στη λεζάντα  αναφέρεται ότι: Για την πρώτη εξεταστική του εαρινού εξαμήνου(Α ΕΑΡ) προσήλθαν  για να εξεταστούν 21 φοιτητές/τριες.
Από αυτούς τους φοιτητές κανείς δεν βαθμολογήθηκε στο διάστημα[0, 1].
Το ποσοστό επιτυχίας ήταν \textit{85.7\%} και η μέση τιμή του βαθμού ήταν \textit{6.5}.\\
Για την δεύτερη  εξεταστική του εαρινού εξαμήνου(Β ΕΑΡ) προσήλθαν για να εξεταστούν 3
φοιτητές/τριες. Από αυτούς τους φοιτητές κανείς δεν βαθμολογήθηκε στο διάστημα[0, 1].Το ποσοστό επιτυχίας ήταν \textit{100\%} και η μέση τιμή του βαθμού ήταν \textit{5.5}.\medskip

Στο Σχήμα \ref{fig:mscexam2015} απεικονίζονται τα στατιστικά  για το ακαδημαϊκό έτος 2015-2016. Στη λεζάντα  αναφέρεται ότι: Για την πρώτη εξεταστική του εαρινού εξαμήνου(Α ΕΑΡ) προσήλθαν  για να εξεταστούν 21 φοιτητές/τριες.
Από αυτούς τους φοιτητές όλοι πέρασαν το μάθημα και η μέση τιμή του βαθμού ήταν 6.7. \medskip

Στο Σχήμα \ref{fig:mscexam2016} απεικονίζονται τα στατιστικά  για το ακαδημαϊκό έτος 2016-2017. Στη λεζάντα  αναφέρεται ότι: Για την πρώτη εξεταστική του εαρινού εξαμήνου(Α ΕΑΡ) προσήλθαν  για να εξεταστούν 20 φοιτητές/τριες.
Από αυτούς ένα ποσοστό \textit{5\%} βαθμολογήθηκε στο διάστημα[0,1].
Το ποσοστό επιτυχίας ήταν \textit{78.9\%} και η μέση τιμή του βαθμού ήταν \textit{5.9}.\\
Για την δεύτερη  εξεταστική του εαρινού εξαμήνου(Β ΕΑΡ) προσήλθαν για να εξεταστούν 5
φοιτητές/τριες. Από αυτούς τους φοιτητές κανείς δεν βαθμολογήθηκε στο διάστημα[0, 1].Το ποσοστό επιτυχίας ήταν \textit{60\%} και η μέση τιμή του βαθμού ήταν \textit{4.2}.

\graphicspath{{./figures/msc/}}
% Density  Plots Per Exam 
\begin{figure}[h]
	\centering
	\begin{subfigure}[b]{0.475\textwidth}
		\centering 
		\includegraphics[width=\textwidth]{mscexam2013}
		\caption{Ακαδημαικό Έτος 2013-2014}
		\label{fig:mscexam2013}
	\end{subfigure}
	\hfill
	\begin{subfigure}[b]{0.475\textwidth}
		\centering 
		\includegraphics[width=\textwidth]{mscexam2014}
		\caption{Ακαδημαικό Έτος 2014-2015}
		\label{fig:mscexam2014}
	\end{subfigure}
	
	\begin{subfigure}[b]{0.475\textwidth}
		\centering 
		\includegraphics[width=\textwidth]{mscexam2015}
		\caption{Ακαδημαικό Έτος 2015-2016}
		\label{fig:mscexam2015}
	\end{subfigure}
	\hfill
	\begin{subfigure}[b]{0.475\textwidth}
		\centering 
		\includegraphics[width=\textwidth]{mscexam2016}
		\caption{Ακαδημαικό Έτος 2016-2017}
		\label{fig:mscexam2016}
	\end{subfigure}
	\caption{Μεταπτυχιακό Μάθημα: Κατανομή συχνότητας βαθμών ανά εξεταστική Περίοδο}	
	\label{fig:mscdensExam}
\end{figure}

\clearpage
Στο Σχήμα \ref{fig:mscyear2013} φαίνεται η συνάθροιση  των ποσοστών επιτυχίας και αποτυχίας στο μεταπτυχιακό μάθημα \textit{Ανάκτηση Πληροφοριών στο Διαδίκτυο} για το ακαδημαϊκό έτος  2013-2014.
Στο σύνολο των εξεταστικών περιόδων του ακαδημαϊκού έτους, βαθμολογήθηκαν 39 γραπτά. Το \textit{5.1\%} των τελευταίων βαθμολογήθηκαν με βαθμό στο διάστημα[0, 1] και δεν λογίζονται στη συνέχεια. Το \textit{75.7\%}των υπόλοιπων γραπτών βαθμολογήθηκαν με προβιβάσιμο βαθμό και η μέση τιμή του βαθμού ήταν \textit{5.4}.\medskip

Στο Σχήμα \ref{fig:mscyear2014} φαίνεται η συνάθροιση  των ποσοστών επιτυχίας και αποτυχίας στο για το ακαδημαϊκό έτος  2014-2015.
Στο σύνολο των εξεταστικών περιόδων του ακαδημαϊκού έτους, βαθμολογήθηκαν 24 γραπτά. Κανένα από τα γραπτά δεν βαθμολογήθηκε στο διάστημα [0, 1]. Το \textit{87.5\%} των  γραπτών βαθμολογήθηκαν με προβιβάσιμο βαθμό και η μέση τιμή του βαθμού ήταν \textit{6.4}.\medskip

Στο Σχήμα \ref{fig:mscyear2015} φαίνεται η συνάθροιση  των ποσοστών επιτυχίας και αποτυχίας στο για το ακαδημαϊκό έτος  2015-2016.
Στο σύνολο των εξεταστικών περιόδων του ακαδημαϊκού έτους, βαθμολογήθηκαν 13 γραπτά. Όλα τα γραπτά βαθμολογήθηκαν με προβιβάσιμο βαθμό και η μέση τιμή του βαθμού ήταν \textit{6.7} . \medskip

Στο Σχήμα \ref{fig:mscyear2016} φαίνεται η συνάθροιση  των ποσοστώνγια το ακαδημαϊκό έτος  2016-2017.
Στο σύνολο των εξεταστικών περιόδων του ακαδημαϊκού έτους, βαθμολογήθηκαν 25 γραπτά. Το \textit{4\%} των τελευταίων βαθμολογήθηκαν με βαθμό στο διάστημα[0, 1] και δεν λογίζονται στη συνέχεια. Το \textit{75\%}των υπόλοιπων γραπτών βαθμολογήθηκαν με προβιβάσιμο βαθμό και η μέση τιμή του βαθμού ήταν \textit{5.5}.
% Μεταπτυχιακό Density Plots Per Year
\begin{figure}[h]
	\centering
	\begin{subfigure}[b]{0.475\textwidth}
		\centering 
		\includegraphics[width=\textwidth]{mscyear2013}
		\caption{Ακαδημαικό Έτος 2013-2014}
		\label{fig:mscyear2013}
	\end{subfigure}
	\hfill
	\begin{subfigure}[b]{0.475\textwidth}
		\centering 
		\includegraphics[width=\textwidth]{mscyear2014}
		\caption{Ακαδημαικό Έτος 2014-2015}
		\label{fig:mscyear2014}
	\end{subfigure}
	
	\begin{subfigure}[b]{0.475\textwidth}
		\centering 
		\includegraphics[width=\textwidth]{mscyear2015}
		\caption{Ακαδημαικό Έτος 2015-2016}
		\label{fig:mscyear2015}
	\end{subfigure}
	\hfill
	\begin{subfigure}[b]{0.475\textwidth}
		\centering 
		\includegraphics[width=\textwidth]{mscyear2016}
		\caption{Ακαδημαικό Έτος 2016-2017}
		\label{fig:mscyear2016}
	\end{subfigure}
	\caption{Μεταπτυχιακό Μάθημα: Κατανομή συχνότητας βαθμών ανά ακαδημαϊκό έτος}	
	\label{fig:mscdensYear}
\end{figure}

\clearpage
% Μεταπτυχιακό  barplots
Το Σχήμα \ref{fig:mscbar2013} αναφέρεται στο μεταπτυχιακό  μάθημα \textit{Ανάκτηση Πληροφοριών στο Διαδίκτυο}. Αφορά φοιτητές/τριες που ξεκίνησαν να εξετάζονται στο εν λόγω μάθημα το ακαδημαϊκό έτος 2013-14. Ξεκινά να τους \textit{παρακολουθεί} από την αρχή του ακαδημαϊκού έτους 2013-2014 έως και την Β' εξεταστική περίοδο του ακαδημαϊκού έτους 2016-17. Τα ποσοστά που αναγράφονται στην πρώτη στήλη του ιστογράμματος διαβάζονται ως εξής: από τους φοιτητές που εξετάστηκαν πρώτη φορά το ακαδημαϊκό έτος 2013-14 στο εν λόγω μάθημα, το 70\% το πέρασε (με την \textit{πρώτη}) το 23.3\% το πέρασε με την \textit{δεύτερη} φορά και το 6.7\% με την \textit{τρίτη} φορά.\medskip

Το Σχήμα \ref{fig:mscbar2014} αφορά φοιτητές/τριες που ξεκίνησαν να εξετάζονται στο εν λόγω μάθημα το ακαδημαϊκό έτος 2014-15. Ξεκινά να τους \textit{παρακολουθεί} από την αρχή του ακαδημαϊκού έτους 2014-2015 έως και την Β' εξεταστική περίοδο του ακαδημαϊκού έτους 2016-17. Τα ποσοστά που αναγράφονται στην πρώτη στήλη του ιστογράμματος διαβάζονται ως εξής: από τους φοιτητές που εξετάστηκαν πρώτη φορά το ακαδημαϊκό έτος 2014-15 στο εν λόγω μάθημα, το 84.2\% των φοιτητών το πέρασε (με την \textit{πρώτη}) φορά, ενώ το 15.8\% το πέρασε με την \textit{δεύτερη} φορά.\medskip

Το Σχήμα \ref{fig:mscbar2015} αφορά φοιτητές/τριες που ξεκίνησαν να εξετάζονται στο εν λόγω μάθημα το ακαδημαϊκό έτος 2015-16. Ξεκινά να τους \textit{παρακολουθεί} από την αρχή του ακαδημαϊκού έτους 2015-2016 έως και την Β' εξεταστική περίοδο του ακαδημαϊκού έτους 2016-17. Τα ποσοστά που αναγράφονται στην πρώτη στήλη του ιστογράμματος διαβάζονται ως εξής: από τους φοιτητές που εξετάστηκαν πρώτη φορά το ακαδημαϊκό έτος 2015-16 στο εν λόγω μάθημα το πέρασαν όλοι οι φοιτητές με την πρώτη φορά.\medskip

Το Σχήμα \ref{fig:mscbar2016} αφορά φοιτητές/τριες που ξεκίνησαν να εξετάζονται στο εν λόγω μάθημα το ακαδημαϊκό έτος 2016-17. Ξεκινά να τους \textit{παρακολουθεί} από την αρχή του ακαδημαϊκού έτους 2016-2017 έως και την Β' εξεταστική περίοδο του ίδιου ακαδημαϊκού έτους. Τα ποσοστά που αναγράφονται στην πρώτη στήλη του ιστογράμματος διαβάζονται ως εξής:από τους φοιτητές που εξετάστηκαν πρώτη φορά το ακαδημαϊκό έτος 2016-17 στο εν λόγω μάθημα, το 75\% των φοιτητών το πέρασε (με την \textit{πρώτη}) φορά, το 15\% το πέρασε με την \textit{δεύτερη }φορά, ενώ ένα ποσοστό 10\% έχει εξεταστεί στο συγκεκριμένο μάθημα δύο φορές αλλά δεν το έχει περάσει.


\begin{figure}[h]
	\centering
	\begin{subfigure}[b]{0.475\textwidth}
		\centering 
		\includegraphics[width=\textwidth]{mscbar2013}
		\caption{Ακαδημαικό Έτος 2013-2014}
		\label{fig:mscbar2013}
	\end{subfigure}
	\hfill
	\begin{subfigure}[b]{0.475\textwidth}
		\centering 
		\includegraphics[width=\textwidth]{mscbar2014}
		\caption{Ακαδημαικό Έτος 2014-2015}
		\label{fig:mscbar2014}
	\end{subfigure}
	
	\begin{subfigure}[b]{0.475\textwidth}
		\centering 
		\includegraphics[width=\textwidth]{mscbar2015}
		\caption{Ακαδημαικό Έτος 2015-2016}
		\label{fig:mscbar2015}
	\end{subfigure}
	\hfill
	\begin{subfigure}[b]{0.475\textwidth}
		\centering 
		\includegraphics[width=\textwidth]{mscbar2016}
		\caption{Ακαδημαικό Έτος 2016-2017}
		\label{fig:mscbar2016}
	\end{subfigure}
	\caption{Μεταπτυχιακό Μάθημα: Προσπάθειες (άξονας \foreignlanguage{english}{Try}) και ποσοστά επιτυχίας/αποτυχίας εξετασθέντων (άξονας \foreignlanguage{english}{Percentage})}	
	\label{fig:mscdensBar}
\end{figure}

\clearpage
\section{Εξόρυξη Κανόνων Συσχέτισης}
\subsection{Σύνολο Δεδομένων}
Το σύνολο δεδομένων που χρησιμοποιήσαμε για αυτό το μέρος της ανάλυσης αποτελείται από 166461 εγγραφές(γραπτά) και 8 μεταβλητές(πληροφορίες για τα γραπτά):
\begin{itemize}
\item \foreignlanguage{english}{ID} Φοιτητή
\item \foreignlanguage{english}{ID} Μαθήματος
\item Τίτλος Μαθήματος
\item Εξεταστική 
\item Ακαδημαϊκό Έτος
\item Βαθμός 
\item Κατεύθυνση Φοιτητή
\item Τύπος Μαθήματος(Θεωρία ή Εργαστήριο)
\end{itemize}
Περιλαμβάνει δεδομένα για τις εξεταστικές περιόδους  από το ακαδημαϊκό έτος 2003-2004 έως και το χειμερινό εξάμηνο του 2017-2018. Σε αυτό το χρονικό  διάστημα  υπάρχουν 3 διαφορετικά προπτυχιακά προγράμματα σπουδών το Π3,Π4 και Π5. Οι φοιτητές  που εξετάστηκαν στα 15 ακαδημαϊκά έτη είναι 3210. Κατά τη διάρκεια των τριών διαφορετικών προγραμμάτων σπουδών υπάρχουν μαθήματα που αφαιρέθηκαν ή άλλαξαν όνομα. Επίσης υπάρχουν μικτά μαθήματα που τους αφαιρέθηκε το εργαστηριακό μέρος. Περισσότερες  λεπτομέρειες για τον  χειρισμό των  παραπάνω περιπτώσεων  υπάρχουν στην επόμενη ενότητα.

\subsection{Καθαρισμός και Προετοιμασία Δεδομένων}
Προεπεξεργαστήκαμε τα δεδομένα ώστε να μπορέσουμε να τα εισάγουμε στον αλγόριθμο εξόρυξης κανόνων. Συγκεκριμένα, προσθέσαμε μία καινούργια μεταβλητή  η οποία  έχει για κάθε μάθημα από το Π3, Π4  το αντίστοιχό τους στο Π5 (Παράρτημα Α). Με τον τρόπο αυτό μαθήματα που έχουν αλλάξει όνομα θα λογίζονται σαν ένα από τον αλγόριθμο. Στη συνέχεια ομαδοποιήσαμε  όλα τα γραπτά ανά φοιτητή και για κάθε ένα από αυτούς πραγματοποιήσαμε την εξής διαδικασία:
\begin{itemize}
\item Κρατήσαμε μόνο τα γραπτά  που βαθμολογήθηκαν με βαθμό μεγαλύτερο ή ίσο του πέντε(δηλαδή ο φοιτητής πέρασε το μάθημα).
\item Επίσης κρατήσαμε τα μαθήματα που είναι μόνο θεωρία και από τα μικτά εκείνα που ο φοιτητής έχει περάσει  και τα δυο μέρη: Θεωρία και Εργαστήριο.
\item Από μαθήματα που ήταν  μικτά σε προηγούμενα προγράμματα σπουδών  και στο Π5 έχει αφαιρεθεί το εργαστήριο, λάβαμε υπόψιν μόνο τη θεωρία.
\item Στη συνέχεια υπολογίσαμε τον τελικό βαθμό για τα μικτά μαθήματα, ο οποίος είναι 60\% θεωρία και 40\% εργαστήριο.
\item Αν ο φοιτητής έδωσε το μάθημα περισσότερες από δύο φορές, αφαιρέσαμε 0,3 από το βαθμό του για κάθε επιπλέον προσπάθεια.
\item Ο βαθμός που προέκυψε από τα προηγούμενα βήματα τον κατηγοριοποιήσαμε ώστε να είναι δυνατή η μετατροπή των δεδομένων σε συναλλαγές(\foreignlanguage{english}{transactions})όπως απαιτεί ο αλγόριθμος. Για την κατηγοριοποίηση έγινε ο μετασχηματισμός των συνεχών τιμών σε  διαβαθμισμένες τιμές  σύμφωνα με το Πίνακα \ref{table:discetize}.
\end{itemize} 


\begin{table}[h]
\centering
\begin{tabular}{|c|c|}
\hline 
Διάστημα Βαθμολογίας & Διαβαθμισμένη Κατηγορία\\ 
\hline 
   [0, 6.5] & μέτρια \\ 
\hline 
[6.6, 7.5] & καλά \\ 
\hline 
[7.7, 8.5] & πολύ καλά \\ 
\hline 
[8.6, 10] & άριστα \\ 
\hline 
\end{tabular}
\caption{Αντιστοίχηση διαβαθμισμένης κατηγορίας με διαστήματα βαθμολογίας} 
\label{table:discetize}
\end{table} \medskip

Στον Πίνακα \ref{table:percent} παρουσιάζεται το ποσοστό των γραπτών ανά την διαβαθμισμένη κατηγορία στην οποία εντάσσονται.
\begin{table}[h]
\centering
\begin{tabular}{|c|c|}
\hline 
Διαβαθμισμένη Κατηγορία & Ποσοστό Γραπτών \\ 
\hline 
μέτρια & 63\% \\ 
\hline 
καλά & 18\% \\ 
\hline 
πολύ καλά & 10\% \\ 
\hline 
άριστα & 7,5\% \\ 
\hline 
\end{tabular}
\caption{Ποσοστό γραπτών ανά διαβαθμισμένη κατηγορία} 
\label{table:percent}
\end{table} 



\subsection{Εξαγωγή Κανόνων Συσχέτισης με τον Αλγόριθμο \foreignlanguage{english}{Apriori}}
	\subsubsection{Ο αλγόριθμος \foreignlanguage{english}{Apriori}}
Ο αλγόριθμος \foreignlanguage{english}{Apriori} \cite{assoc_agrawal} είναι ένας από τους πιο γνωστούς αλγορίθμους για την εξόρυξη κανόνων συσχέτισης. Βασική δομή στην οποία στηρίζεται ο αλγόριθμος είναι τα στοιχειοσύνολα.  Ο \foreignlanguage{english}{Apriori} βασίζεται στην αρχή ότι όλα τα υποσύνολα ενός στοιχειοσυνόλου που εμφανίζεται συχνά  , θα είναι και αυτά με την σειρά τους συχνά.
Συγκεκριμένα, έστω συχνό στοιχειοσύνολο \{Α,Β,Γ,Δ\} με βάση την αρχή που στηρίζεται ο \foreignlanguage{english}{Apriori}, τότε  και τα υποσύνολα \{Α\}, \{Β\}, \{Γ\}, \{Δ\} θα είναι επίσης συχνά. Επίσης για κάθε μη συχνό υποσύνολο, πρέπει όλα τα υπερσύνολα στα οποία περιέχεται να είναι και αυτά μη συχνά.  \bigskip

Βασικά μεγέθη του \foreignlanguage{english}{Apriori} είναι η υποστήριξη(\foreignlanguage{english}{support}), εμπιστοσύνη(\foreignlanguage{english}{confidence}) και ανύψωση(\foreignlanguage{english}{lift}). Συγκεκριμένα, έστω \foreignlanguage{english}{X}, \foreignlanguage{english}{Y} τυχαία στοιχειοσύνολα, η υποστήριξη(\foreignlanguage{english}{X}) ισούται με τον αριθμό των συναλλαγών στις οποίες υπάρχει το  \foreignlanguage{english}{X} δια τον συνολικό αριθμό των συναλλαγών.\bigskip

Η εμπιστοσύνη ορίζεται ως:
$confidence (X \to Y) = \frac{support (X \cup Y)}{confidence (X)}$ και δηλώνει πόσο πιθανό είναι να επιλεχθεί το στοιχειοσύνολο $Y$ δεδομένου ότι έχει επιλεχθεί πρώτα το στοιχειοσύνολο $X$.\bigskip

Η ανύψωση ορίζεται ως:
$lift(X \to Y) = \frac{support(X \cup Y)}{support(X) * support(Y)}  $ και δηλώνει  πόσο πιθανό είναι να επιλεχθεί το στοιχειοσύνολο $Y$  δεδομένου ότι έχει επιλεχθεί το στοιχειοσύνολο $X$ λαμβάνοντας υπόψη και το πόσο συχνά εμφανίζεται στις συναλλαγές το  $Y$. \bigskip

Η διαδικασία εξόρυξης κανόνων με τον \foreignlanguage{english}{Apriori} χωρίζεται σε δύο κυρίως βήματα: 
\begin{enumerate}
	\item Εύρεση των συχνών στοιχειοσυνόλων(δηλαδή εκείνα που ικανοποιούν την ελάχιστη υποστήριξη).
	\item Εξόρυξη συχνών και αξιόπιστων κανόνων συσχέτισης.
	\begin{enumerate}
	\item Εξόρυξη κανόνων από συχνά στοιχειοσύνολα.
	\item Επιλογή αυτών που η εμπιστοσύνη είναι μεγαλύτερη ή ίση από εκείνη τις ελάχιστης προκαθορισμένης.
	\end{enumerate}
\end{enumerate}

Για παράδειγμα, έστω ότι έχουμε μια βάση δεδομένων με Χ συναλλαγές. Ο αλγόριθμός διαβάζει τον πίνακα με τις συναλλαγές. Ο πίνακας θα διαβαστεί το πολύ όσες φορές είναι και ο αριθμός τον στοιχείων που περιέχει. Στο πρώτο πέρασμα του πίνακα ο αλγόριθμος υπολογίζει την υποστήριξη των στοιχειοσυνόλων, έπειτα κρατάει αυτά που ικανοποιούν την προκαθορισμένη ελάχιστη υποστήριξη. Σε κάθε επόμενο βήμα χρησιμοποιούνται τα στοιχειοσύνολα του προηγουμένου περάσματος για να δημιουργηθούν καινούργια υποψήφια συχνά στοιχειοσύνολα. Στο τέλος κάθε βήματος υπολογίζεται ποια στοιχειοσύνολα πληρούν την προκαθορισμένη υποστήριξη  έτσι ώστε να χρησιμοποιηθούν στο επόμενο βήμα. Λαμβάνοντας υπόψη την βασική αρχή του \foreignlanguage{english}{Apriori} απορρίπτεται ένας μεγάλος αριθμός στοιχειοσυνόλων τα  οποία είναι υπερσύνολα μη συχνών συνόλων. Η παραπάνω διαδικασία τερματίζεται όταν δεν υπάρχει πλέον κάποιο στοιχειοσύνολο που να ικανοποιεί τον περιορισμό της ελάχιστης υποστήριξης. Στο Σχήμα \ref{fig:apriori} φαίνεται το διάγραμμα ροής τους αλγορίθμου. \\

\graphicspath{{./arules/}}
\begin{figure}[h]
         \centering
         \includegraphics[width=\textwidth]{aprioriAlg}
         \caption{Διάγραμμα Ροής Αλγορίθμου}
         \label{fig:apriori}
\end{figure}
     
Παρακάτω παρουσιάζεται η διαδικασία του \foreignlanguage{english}{Apriori} σε μορφή ψευδοκώδικα:\\
\selectlanguage{english}
\begin{algorithm}
\caption{Apriori}
\begin{algorithmic}

\STATE $C_{k}$: Candidate itemset of size k 
\STATE $L_{k}$: frequent itemset of size k
\STATE $L_{1} = $ \{frequent items\}
\FOR{$k=1; L_{k} \neq 0; k++;$}
\STATE $C_{k+1} =$ candidates generated from $L_{k}$
\FORALL{transaction $t$ in database}
\STATE Increment the count of all candidates in $C_{k+1}$
\STATE Those are contained in $t$
\ENDFOR
\STATE  $L_{k+1} =$ candidates in $C_{k+1}$ with min\_support
\ENDFOR
\RETURN $U_{k} L_{k}$
\end{algorithmic}
\end{algorithm}
\selectlanguage{greek}
\paragraph{}
Τέλος ο \foreignlanguage{english}{Apriori} παρουσιάζει δύο σημαντικά μειονεκτήματα: πρώτων η παραγωγή μεγάλου αριθμού κανόνων χωρίς ενδιαφέρον και δεύτερον η παραγωγή μεγάλου μήκους στοιχειοσυνόλων όπως αναλυτικά αναφέρει ο \foreignlanguage{english}{Hegland}[2].\\



\subsubsection{Εφαρμογή του \foreignlanguage{english}{Apriori}}

 Η διαδικασία που ακολουθήσαμε  για την εξόρυξη κανόνων συνοψίζεται στα παρακάτω βήματα:\\
 \begin{enumerate}
 \item Ρύθμιση παραμέτρων (\foreignlanguage{english}{support}, \foreignlanguage{english}{conf}, \foreignlanguage{english}{maxlen} ).
 \item Εισαγωγή συναλλαγών στον \foreignlanguage{english}{Apriori} με τις προκαθορισμένες παραμέτρους.
 \item Αφαίρεση  πλεοναζόντων κανόνων (\foreignlanguage{english}{pruning}).
 \item Φιλτράρισμα και συλλογή κανόνων που μας ενδιαφέρουν.
 \end{enumerate}
   
\paragraph{}
Πρώτο βήμα είναι η ρύθμιση των παραμέτρων του \foreignlanguage{english}{Apriori}. Ορίσαμε το \foreignlanguage{english}{support} και το \foreignlanguage{english}{condidence} σε διάφορες τιμές. Όσο μικρότερες είναι αυτές οι τιμές τόσο περισσότεροι είναι οι κανόνες που παράγει ο \foreignlanguage{english}{Apriori}. Ο αριθμός των κανόνων ανάλογα με  αυτές τις παραμέτρους μπορεί να κυμαίνεται από μερικές χιλιάδες μέχρι κάποια εκατομμύρια οι περισσότεροι εκ των οποίων είναι θόρυβος(χαμηλού ενδιαφέροντος). Η επιλογή τιμών πολύ κοντά στο μηδέν (στο \foreignlanguage{english}{support}) οδηγούν  σε τεράστια σύνολα παραγόμενων κανόνων τα οποία είναι δύσκολο να χειριστεί ένας συμβατικός ηλεκτρονικός υπολογιστής.\\
 Ενδεικτικά στον πίνακα \ref{table:rule_stats} οι αριθμοί  των κανόνων που παρήχθησαν  για διάφορες τιμές των \foreignlanguage{english}{support}, \foreignlanguage{english}{confidence} και \foreignlanguage{english}{maxlen}:
\begin{table}[h]
\centering
\begin{tabular}{|l|l|l|l|}
\hline
\foreignlanguage{english}{maxlen} & \foreignlanguage{english}{support} & \foreignlanguage{english}{confidence} & \foreignlanguage{english}{rules} \\ \hline
10 & 0.01 & 0.55 & \foreignlanguage{english}{out of memory} \\ \hline
5 & 0.01 & 0.55 & 787814 \\ \hline
7 & 0.01 & 0.6 & 4118656 \\ \hline
7 & 0.01 & 0.5 & 4596925 \\ \hline
\end{tabular}
\caption{Αριθμός κανόνων για διάφορες τιμές των \foreignlanguage{english}{maxlen, support} και $confidence$. }
\label{table:rule_stats}
\end{table}
\clearpage

\paragraph{} 
 Επόμενο βήμα ήταν η εισαγωγή των δεδομένων στον \foreignlanguage{english}{Apriori} με τις κατάλληλες παραμέτρους. Σκοπός ήταν να βρούμε σε ποια μαθήματα οι φοιτητές τείνουν να έχουν υψηλή βαθμολογία δεδομένου ότι τα έχουν πάει καλά σε κάποιο συγκεκριμένο μάθημα. Το 63\% των γραπτών είχαν ως ανηγμένο  βαθμό \textit{μέτρια}. Λαμβάνοντας υπόψιν αυτό το γεγονός επιλέξαμε πολύ μικρό \foreignlanguage{english}{support}(0,005) ώστε να μπορέσει  ο αλγόριθμος να εντοπίσει κανόνες για επίδοση \textit{καλά}, \textit{πολύ καλά} και \textit{άριστα}. \foreignlanguage{english}{condidence} ορίσαμε 0,5 ώστε να επιλέξουμε εμείς τους κανόνες με βάση το \foreignlanguage{english}{lift}.
\paragraph{}
Στη συνέχεια αφαιρέσαμε τους πλεονάζοντες κανόνες. Όπως αναφέρει ο \foreignlanguage{english}{Bayardo} (200) 'Πλεονάζον είναι ένας κανόνας εάν υπάρχει ένας πιο γενικός κανόνας με ίδιο ή μεγαλύτερο \foreignlanguage{english}{confidence} . Συνεπώς, ένας πιο συγκεκριμένος κανόνας είναι πλεονάζον εάν είναι μόνο ίσος ή λιγότερο προγνωστικός  από έναν πιο γενικό κανόνα. Ένας κανόνας είναι πιο γενικός αν έχει το ίδιο δεξιό μέρος και ένα ή περισσότερα στοιχεία λιγότερα στο αριστερό μέρος'. \\
 Για την αφαίρεση των πλεοναζόντων κανόνων στο \foreignlanguage{english}{RStudio} χρησιμοποιήσαμε την μέθοδο \foreignlanguage{english}{\emph{is.redundant()}} από το πακέτο \foreignlanguage{english}{\textit{arules}}
\paragraph{}
Τελευταίο βήμα ήταν το φιλτράρισμα  των κανόνων ώστε να αφαιρέσουμε  τον θόρυβο  και κανόνες που δεν μας ενδιαφέρουν. Σκοπός της εργασίας ήταν να βρούμε κανόνες για μαθήματα στα οποία οι φοιτητές τείνουν να έχουν υψηλές επιδόσεις. Η πρώτη ενέργεια ήταν να βρούμε μόνο τους κανόνες από τους οποίους το δεξιό μέρος περιλάμβανε κάποιο μάθημα το οποίο ο φοιτητής είχε \textit{καλή}, \textit{πολύ καλή} ή \textit{άριστη} επίδοση. Στη συνέχεια από αυτούς τους κανόνες  κρατήσαμε εκείνους που είχαν  περισσότερες  από δέκα εμφανίσεις(\foreignlanguage{english}{count}) στις συναλλαγές και \foreignlanguage{english}{lift} μεγαλύτερο ή ίσο του $4$ καθώς θεωρήσαμε ότι κανόνες με μικρότερες τιμές θα είναι αμφιβόλου αξίας.

\subsection{Αποτελέσματα}

 % arules Scatter plots 

Το Σχήμα \ref{fig:r_good} μας πληροφορεί ότι: ο αλγόριθμος(\foreignlanguage{english}{Apriori}) εξόρυξε \textit{3549} κανόνες συσχέτισης για στοιχειοσύνολα που στο δεξί τους μέλος έχουν μαθήματα με επίδοση \textit{καλά} (δηλαδή κάθε κουκίδα είναι ένας κανόνας της μορφής: \\

\{ ... \} $\implies$\{μάθημα $\to$ \textbf{καλά}\} \\

Ο άξονας $x$ μας πληροφορεί για το \foreignlanguage{english}{support} των κανόνων και ο άξονας $y$ για το \foreignlanguage{english}{confidence}. Στο δεξιό μέρος του σχήματος υπάρχει η στήλη που έχει την πληροφορία για την τιμή του \foreignlanguage{english}{lift} (όσο πιο κόκκινο το χρώμα της κουκίδας τόσο μεγαλύτερη η τιμή του \foreignlanguage{english}{lift}). Στο σχήμα παρατηρούμε ότι όσο μικρότερο είναι το  \foreignlanguage{english}{support} του κανόνα τόσο μεγαλύτερο είναι το \foreignlanguage{english}{lift}.\medskip

Το Σχήμα \ref{fig:r_great} μας πληροφορεί ότι: ο αλγόριθμος(\foreignlanguage{english}{Apriori}) εξόρυξε \textit{230} κανόνες συσχέτισης για στοιχειοσύνολα που στο δεξί τους μέλος έχουν μαθήματα με επίδοση \textit{πολύ καλά} (δηλαδή κάθε κουκίδα είναι ένας κανόνας της μορφής: \\

\{ ... \} $ \implies $ \{μάθημα $\to$ \textbf{πολύ καλά}\} \\

Ο άξονας $x$ μας πληροφορεί για το \foreignlanguage{english}{support} των κανόνων και ο άξονας $y$ για το \foreignlanguage{english}{confidence}. Στο δεξιό μέρος του σχήματος υπάρχει η στήλη που έχει την πληροφορία για την τιμή του \foreignlanguage{english}{lift} (όσο πιο κόκκινο το χρώμα της κουκίδας τόσο μεγαλύτερη η τιμή του \foreignlanguage{english}{lift}).\medskip

 Αντίστοιχα και για το Σχήμα \ref{fig:r_best} ο αλγόριθμος(\foreignlanguage{english}{Apriori}) εξόρυξε \textit{684} κανόνες συσχέτισης για στοιχειοσύνολα που στο δεξί τους μέλος έχουν μαθήματα με επίδοση \textit{άριστα}(δηλαδή κάθε κουκίδα είναι ένας κανόνας της μορφής: \\

\{ ... \} $\implies$ \{μάθημα $\to$ \textbf{άριστα}\} \\

Ο άξονας $x$ μας πληροφορεί για το \foreignlanguage{english}{support} των κανόνων και ο άξονας $y$ για το \foreignlanguage{english}{confidence}. Στο δεξιό μέρος του σχήματος υπάρχει η στήλη που έχει την πληροφορία για την τιμή του \foreignlanguage{english}{lift}(όσο πιο κόκκινο το χρώμα της κουκίδας τόσο μεγαλύτερη η τιμή του \foreignlanguage{english}{lift}). Στο σχήμα παρατηρούμε ότι όσο μικρότερο είναι το  \foreignlanguage{english}{confidence} του κανόνα τόσο μεγαλύτερο είναι το \foreignlanguage{english}{lift}.\medskip




 \selectlanguage{english}
\begin{lstlisting}[language=R]
#  Scatter Plot Rules
library('arulesViz')
plot(rules_good, measure = c("support", "confidence"), 
	shading = "lift")
plot(rules_great, measure = c("support", "confidence"), 
	shading = "lift")
plot(rules_best, measure = c("support", "confidence"), 
	shading = "lift")
\end{lstlisting}
\selectlanguage{greek}
\graphicspath{{./arules/}}
\begin{figure}[h]
     \centering
     \begin{subfigure}[b]{0.475\textwidth}
         \centering
         \includegraphics[width=\textwidth]{rules_good1}
         \caption{Κανόνες: Δεξί Μέλος \textit{καλά}}
         \label{fig:r_good}
     \end{subfigure}
     \hfill
     \begin{subfigure}[b]{0.475\textwidth}
         \centering
         \includegraphics[width=\textwidth]{rules_great1}
         \caption{Κανόνες: Δεξί Μέλος \textit{πολύ καλά}}
         \label{fig:r_great}
     \end{subfigure}
     
     \begin{subfigure}[b]{0.475\textwidth}
         \centering
         \includegraphics[width=\textwidth]{rules_best1}
         \caption{Κανόνες: Δεξί Μέλος \textit{άριστα}}
         \label{fig:r_best}
     \end{subfigure}
        \label{fig:scatter_rules}
       	\caption{Γράφημα διασποράς για διάφορες τιμές των $support, confidence$ και $lift$}
\end{figure}
\clearpage 

% Arules Two-Key plots
Το Σχήμα \ref{fig:tk_good} μας δίνει πληροφορίες για τις τιμές των \foreignlanguage{english}{support} και \foreignlanguage{english}{confidence} των παραγόμενων κανόνων όπου στο δεξί τους μέλος υπάρχει μάθημα με επίδοση \textit{καλά} ανάλογα με το μήκος τους (αριθμός στοιχειών που αποτελούνται). Με κόκκινο χρώμα είναι οι κανόνες με μήκος πέντε (5), με γαλάζιο οι κανόνες με μήκος τέσσερα   (4) και με μοβ οι κανόνες με μήκος τρία (3). \medskip

Το Σχήμα \ref{fig:tk_great} μας δίνει πληροφορίες για τις τιμές των \foreignlanguage{english}{support} και \foreignlanguage{english}{confidence} των παραγόμενων κανόνων όπου στο δεξί τους μέλος υπάρχει μάθημα με επίδοση \textbf{πολύ καλά} ανάλογα με το μήκος τους (αριθμός στοιχειών που αποτελούνται).Με κόκκινο χρώμα είναι οι κανόνες με μήκος πέντε(5), με γαλάζιο οι κανόνες με μήκος τέσσερα (4) και με μοβ οι κανόνες με μήκος τρία (3). \medskip

Το Σχήμα \ref{fig:tk_best} μας δίνει πληροφορίες για τις τιμές των \foreignlanguage{english}{support} και \foreignlanguage{english}{confidence} των παραγόμενων κανόνων όπου στο δεξί τους μέλος υπάρχει μάθημα με επίδοση \textbf{άριστα} ανάλογα με το μήκος τους (αριθμός στοιχειών που αποτελούνται). Με κόκκινο χρώμα είναι οι κανόνες με μήκος πέντε (5), με ανοιχτό πράσινο κανόνες με μήκος τέσσερα (4), με γαλάζιο οι κανόνες με μήκος τρία (3) και με μοβ οι κανόνες με μήκος δύο (2).

 \selectlanguage{english}
\begin{lstlisting}[language=R]
# Plots number of items contained in the rule
library('arulesViz')
plot(rules_good, method = "two-key plot")
plot(rules_great, method = "two-key plot")
plot(rules_best, method = "two-key plot")
\end{lstlisting}
\selectlanguage{greek}
\begin{figure}[h]
     \centering
     \begin{subfigure}[b]{0.475\textwidth}
         \centering
         \includegraphics[width=\textwidth]{tk_good1}
         \caption{Κανόνες: Δεξί Μέλος \textit{καλά}}
         \label{fig:tk_good}
     \end{subfigure}
     \hfill
     \begin{subfigure}[b]{0.475\textwidth}
         \centering
         \includegraphics[width=\textwidth]{tk_great1}
         \caption{Κανόνες: Δεξί Μέλος \textit{πολύ καλά}}
         \label{fig:tk_great}
     \end{subfigure}
     
     \begin{subfigure}[b]{0.475\textwidth}
         \centering
         \includegraphics[width=\textwidth]{tk_best1}
         \caption{Κανόνες: Δεξί Μέλος \textit{άριστα}}
         \label{fig:tk_best}
     \end{subfigure}
        \label{fig:tk_rules}
       	\caption{Γράφημα διασποράς για διάφορες τιμές $support, confidence$ ανά μήκος κανόνα}
\end{figure}
\clearpage
\begin{landscape}
Παρακάτω παρουσιάζονται οι 10 (από τους συνολικούς 133 κανόνες) ισχυρότεροι κανόνες με βάση το \foreignlanguage{english}{lift} οι οποίοι στο δεξί τους μέλος έχουν μάθημα με επίδοση καλά. Το \foreignlanguage{english}{support, confidence} και \foreignlanguage{english}{lift} παρουσιάζονται αναλυτικά στον Πίνακα \ref{table_good}.
   
\begin{enumerate}

    \item \{Αρχές Σχεδίασης Λειτουργικών Συστημάτων $=$ πολύ καλά,Διακριτά Μαθηματικά $=$ μέτρια,Πληροφοριακά Συστήματα ΙΙ $=$ άριστα\} $\Rightarrow$ \{Μηχανική Μάθηση $=$ \textbf{καλά} \}
    
    \item \{Διακριτά Μαθηματικά $=$ μέτρια,Δομές Δεδομένων και Ανάλυση Αλγορίθμων $=$ καλά,Μεθοδολογίες Προγραμματισμού $=$ μέτρια,Συστήματα Διαχείρισης Βάσεων Δεδομένων $=$ καλά\} $\Rightarrow$ \{Ανάπτυξη Διαδικτυακών Συστ. και Εφαρμογών $=$ καλά\}


    \item \{Γλώσσες και Τεχνολογίες Ιστού $=$ μέτρια,Διακριτά Μαθηματικά $=$ μέτρια,Δίκτυα Η/Υ $=$ μέτρια,Τεχνητή Νοημοσύνη: Γλώσσες και Τεχνικές $=$ πολύ καλά\} $\Rightarrow$ \{Αλγοριθμική και Προγραμματισμός $=$ πολύ καλά\}


    \item \{Αρχές Σχεδίασης Λειτουργικών Συστημάτων $=$ μέτρια,Πληροφοριακά Συστήματα Ι $=$ μέτρια,Τεχνητή Νοημοσύνη: Γλώσσες και Τεχνικές $=$ μέτρια,
Τεχνολογία Βάσεων Δεδομένων $=$ πολύ καλά\} $\Rightarrow$ \{Μεθοδολογίες Προγραμματισμού $=$ πολύ καλά\}


    \item \{Αρχές Σχεδίασης Λειτουργικών Συστημάτων $=$ μέτρια,Γλώσσες και Τεχνολογίες Ιστού $=$ μέτρια,Πληροφοριακά Συστήματα Ι $=$ μέτρια,Τεχνητή Νοημοσύνη: Γλώσσες και Τεχνικές $=$ άριστα\} $\Rightarrow$ \{Δομές Δεδομένων και Ανάλυση Αλγορίθμων $=$ πολύ καλά\}


    \item \{Μαθηματική Ανάλυση και Γραμμική Άλγεβρα $=$ μέτρια,Μεθοδολογίες Προγραμματισμού $=$ μέτρια,Συστήματα Διαχείρισης Βάσεων Δεδομένων $=$ καλά,Τεχνολογία Βάσεων Δεδομένων $=$ καλά\}  $\Rightarrow$ \{Εισαγωγή στην Πληροφορική $=$ πολύ καλά\}


    \item \{Διακριτά Μαθηματικά $=$ μέτρια,Πληροφοριακά Συστήματα Ι $=$ μέτρια,Τεχνητή Νοημοσύνη: Γλώσσες και Τεχνικές $=$ άριστα,Τηλεπικοινωνίες και Δίκτυα Υπολογιστών $=$ μέτρια\} $\Rightarrow$\{Δομές Δεδομένων και Ανάλυση Αλγορίθμων $=$ πολύ καλά\}


    \item \{Πληροφοριακά Συστήματα ΙΙ $=$ μέτρια,Τεχνητή Νοημοσύνη: Γλώσσες και Τεχνικές $=$ μέτρια,  Τεχνολογία Βάσεων Δεδομένων $=$ πολύ καλά\} $\Rightarrow$ \{Μεθοδολογίες Προγραμματισμού $=$ πολύ καλά\}

 
    \item \{Αρχές Σχεδίασης Λειτουργικών Συστημάτων $=$ μέτρια,Τεχνητή Νοημοσύνη: Γλώσσες και Τεχνικές $=$ μέτρια,  Τεχνολογία Βάσεων Δεδομένων $=$ πολύ καλά\} $\Rightarrow$  \{Μεθοδολογίες Προγραμματισμού $=$ πολύ καλά\}

    \item \{Αλγοριθμική και Προγραμματισμός $=$ πολύ καλά,Επιχειρησιακή Έρευνα $=$ μέτρια,Τεχνητή Νοημοσύνη: Γλώσσες και Τεχνικές $=$ καλά\} $\Rightarrow$ \{Μεθοδολογίες Προγραμματισμού $=$ πολύ καλά\}
\end{enumerate}
  
  
Παρακάτω παρουσιάζονται οι 10 (συνολικοί κανόνες 13) ισχυρότεροι κανόνες με βάση το \foreignlanguage{english}{lift} οι οποίοι στο δεξί τους μέλος έχουν μάθημα με επίδοση πολύ καλά. Το \foreignlanguage{english}{support, confidence} και \foreignlanguage{english}{lift} παρουσιάζονται αναλυτικά στον Πίνακα \ref{table_great}.
\begin{enumerate}

\item \{Γλώσσες και Τεχνολογίες Ιστού $=$ μέτρια,Διακριτά Μαθηματικά $=$ μέτρια,Δίκτυα Η/Υ $=$ μέτρια,Τεχνητή Νοημοσύνη: Γλώσσες και Τεχνικές $=$ πολύ καλά\} $\Rightarrow$ \{Αλγοριθμική και Προγραμματισμός $=$ πολύ καλά\}


\item \{Αρχές Σχεδίασης Λειτουργικών Συστημάτων $=$ μέτρια,Πληροφοριακά Συστήματα Ι $=$ μέτρια,Τεχνητή Νοημοσύνη: Γλώσσες και Τεχνικές $=$ μέτρια,Τεχνολογία Βάσεων Δεδομένων $=$ πολύ καλά\} $\Rightarrow$ \{Μεθοδολογίες Προγραμματισμού $=$ πολύ καλά\}


\item \{Αρχές Σχεδίασης Λειτουργικών Συστημάτων $=$ μέτρια,Γλώσσες και Τεχνολογίες Ιστού $=$ μέτρια,Πληροφοριακά Συστήματα Ι $=$ μέτρια,Τεχνητή Νοημοσύνη: Γλώσσες και Τεχνικές $=$ άριστα\} $\Rightarrow$ \{Δομές Δεδομένων και Ανάλυση Αλγορίθμων $=$ πολύ καλά\}


\item \{Μαθηματική Ανάλυση και Γραμμική Άλγεβρα $=$ μέτρια,Μεθοδολογίες Προγραμματισμού $=$ μέτρια,Συστήματα Διαχείρισης Βάσεων Δεδομένων $=$ καλά,Τεχνολογία Βάσεων Δεδομένων $=$ καλά\} $\Rightarrow$ \{Εισαγωγή στην Πληροφορική  $=$ πολύ καλά\}

\item \{Διακριτά Μαθηματικά $=$ μέτρια,Πληροφοριακά Συστήματα Ι $=$ μέτρια,Τεχνητή Νοημοσύνη: Γλώσσες και Τεχνικές $=$ άριστα,Τηλεπικοινωνίες και Δίκτυα Υπολογιστών $=$ μέτρια\} $\Rightarrow$ \{Δομές Δεδομένων και Ανάλυση Αλγορίθμων $=$ πολύ καλά\}


\item   \{Πληροφοριακά Συστήματα ΙΙ $=$ μέτρια,Τεχνητή Νοημοσύνη:   Γλώσσες κ Τεχνικές $=$ μέτρια, Τεχνολογία Βάσεων Δεδομένων $=$ πολύ καλά\} $\Rightarrow$ \{Μεθοδολογίες Προγραμματισμού $=$ πολύ καλά\}


\item \{Αρχές Σχεδίασης Λειτουργικών Συστημάτων $=$ μέτρια,Τεχνητή Νοημοσύνη: Γλώσσες και Τεχνικές  $=$ μέτρια,Τεχν\-ολογία Βάσεων Δεδομένων  $=$ πολύ καλά\} $\Rightarrow$ \{Μεθοδολογίες Προγραμματισμού  $=$ πολύ καλά\}


\item \{Αλγοριθμική και Προγραμματισμός $=$ πολύ καλά,Επιχειρησιακή Έρευνα $=$ μέτρια,Τεχνητή Νοημοσύνη: Γλώσσες και Τεχνικές $=$ καλά\} $\Rightarrow$ \{Μεθοδολογίες Προγραμματισμού $=$ πολύ καλά\}


\item \{Δίκτυα Η/Υ $=$ μέτρια,Εισαγωγή στα Λειτουργικά Συστηματα $=$ άριστα,Πληροφοριακά Συστήματα ΙΙ $=$ μέτρια\} $\Rightarrow$ \{Αλληλεπίδραση Ανθρώπου-Μηχανής και Ανάπτυξη Διεπιφανεών Χρήστη $=$ πολύ καλά\}

\item \{Αλληλεπίδραση Ανθρώπου-Μηχανής και Ανάπτυξη Διεπιφανεών Χρήστη $=$ καλά,Αριθ. Ανάλυση και Προγρ/μός Επιστ. Εφαρμογών $=$ καλά,Δομές Δεδομένων και Ανάλυση Αλγορίθμων $=$ μέτρια,Εισαγωγή στα Λειτουργικά Συστηματα $=$ μέτρια\} $\Rightarrow$ \{Δεξιότητες Επικοινωνίας/Κοινωνικά Δίκτυα $=$ πολύ καλά \}

\end{enumerate}
Παρακάτω παρουσιάζονται οι 10 (συνολικοί κανόνες 133) ισχυρότεροι κανόνες με βάση το \foreignlanguage{english}{lift} οι οποίοι στο δεξί τους μέλος έχουν μάθημα με επίδοση άριστα. Το \foreignlanguage{english}{support, confidence} και \foreignlanguage{english}{lift} παρουσιάζονται αναλυτικά στον Πίνακα \ref{table_best}.
\begin{enumerate}

\item \{Μηχανική Μάθηση $=$ άριστα,Τεχνητή Νοημοσύνη: Γλώσσες και Τεχνικές $=$ άριστα\} $\Rightarrow$ \{Αντικειμενοστραφής Προγραμματισμός $=$ άριστα\}
   

\item \{Αριθ. Ανάλυση και Προγρ/μός Επιστ. Εφαρμογών $=$ άριστα,Τεχνητή Νοημοσύνη: Γλώσσες και Τεχνικές $=$ άριστα\} $\Rightarrow$ \{Τεχνολογία Βάσεων Δεδομένων $=$ άριστα\}


\item \{Αριθ. Ανάλυση και Προγρ/μός Επιστ. Εφαρμογών $=$ άριστα,Τεχνητή Νοημοσύνη: Γλώσσες και Τεχνικές $=$ άριστα\} $\Rightarrow$ \{Αντικειμενοστραφής Προγραμματισμός $=$ άριστα\}


\item \{Αλγοριθμική και Προγραμματισμός $=$ άριστα,Ευφυή Συστήματα $=$ άριστα\} $\Rightarrow$ \{Αντικειμενοστραφής Προγραμματισμός  $=$ άριστα\}


\item \{Μεθοδολογίες Προγραμματισμού $=$ άριστα,Πληροφοριακά Συστήματα ΙΙ $=$ άριστα\} $\Rightarrow$ \{Αριθ. Ανάλυση και Προγρ/μός Επιστ. Εφαρμογών $=$ άριστα\}


\item \{Αριθ. Ανάλυση και Προγρ/μός Επιστ. Εφαρμογών $=$ άριστα,Τεχνητή Νοημοσύνη: Γλώσσες και Τεχνικές $=$ άριστα\} $\Rightarrow$ \{Δομές Δεδομένων και Ανάλυση Αλγορίθμων $=$ άριστα\}


\item \{Μεθοδολογίες Προγ. $=$ άριστα,Τεχνολογία Βάσεων Δεδομένων $=$ άριστα\} $\Rightarrow$ \{Δομές Δεδομένων και Ανάλυση Αλγορίθμων $=$ άριστα\}

 
\item \{Εισαγωγή στα Λειτουργικά Συστηματα $=$ άριστα,Τεχνητή Νοημοσύνη: Γλώσσες και Τεχνικές $=$ άριστα\} $\Rightarrow$ \{Τεχνολογία Βάσεων Δεδομένων $=$ άριστα\}


\item \{Μεθοδολογίες Προγραμματισμού $=$ άριστα,Τεχνητή Νοημοσύνη: Γλώσσες και Τεχνικές $=$ άριστα\} $\Rightarrow$   \{ Αντικειμενοστραφής Προγραμματισμός $=$ άριστα\}


\item \{Δομές Δεδομένων και Ανάλυση Αλγορίθμων $=$ άριστα,Εισαγωγή στα Λειτουργικά Συστηματα $=$ άριστα\} $\Rightarrow$ \{Μεθοδολογίες Προγραμματισμού $=$ άριστα\}



\end{enumerate}

\end{landscape}


\clearpage
Ο Πίνακας \ref{table_good}, \ref{table_great} και \ref{table_best} απεικονίζουν τις τιμές των \foreignlanguage{english}{support, confidence, lift} για τους δέκα ισχυρότερους κανόνες που εξόρυξε ο $Apriori$. Οι κανόνες είναι ταξινομημένοι κατά φθίνουσα σειρά με βάση το $lift$. Η στήλη $count$ περιέχει τον αριθμό των συνολικών εμφανίσεων του κανόνα.
Συγκεκριμένα ο πίνακας \ref{table_good} περιέχει τους κανόνες όπου το δεξί τους μέρος περιέχει κάποιο μάθημα με επίδοση 'καλά'. Ο πίνακας \ref{table_great} περιέχει τους κανόνες όπου το δεξί τους μέρος περιέχει κάποιο μάθημα με επίδοση 'πολύ καλά'. Ο πίνακας \ref{table_best} περιέχει τους κανόνες όπου το δεξί τους μέρος περιέχει κάποιο μάθημα με επίδοση 'άριστα'.



\begin{table}
\centering
\caption{Οι δέκα (10) ισχυρότεροι κανόνες για επίδοση \textit{καλά}}
\label{table_good}
\begin{tabular}{|l|l|l|l|} 
\hline
\textbf{$support$} & \textbf{$confidence$} & \textbf{$lift$} & \textbf{$count$}  \\ 
\hline
0.00862069       & 0.714285714         & 9.206349206   & 10              \\ 
\hline
0.00862069       & 0.625               & 8.055555556   & 10              \\ 
\hline
0.00862069       & 0.555555556         & 8.055555556   & 10              \\ 
\hline
0.00862069       & 0.555555556         & 7.581699346   & 10              \\ 
\hline
0.00862069       & 0.769230769         & 7.561929596   & 10              \\ 
\hline
0.00862069       & 0.555555556         & 7.323232323   & 10              \\ 
\hline
0.00862069       & 0.555555556         & 7.323232323   & 10              \\ 
\hline
0.009482759      & 0.55                & 7.25          & 11              \\ 
\hline
0.009482759      & 0.733333333         & 7.209039548   & 11              \\ 
\hline
0.012068966      & 0.7                 & 6.881355932   & 14              \\
\hline
\end{tabular}
\end{table}

\begin{table}
\centering
\caption{Οι δέκα (10) ισχυρότεροι κανόνες για επίδοση \textit{πολύ καλά}}
\label{table_great}
\begin{tabular}{|l|l|l|l|} 
\hline
\textbf{$support$} & \textbf{$confidence$} & \textbf{$lift$} & \textbf{$count$}  \\ 
\hline
0.00862069       & 0.555555556         & 7.581699346   & 10              \\ 
\hline
0.00862069       & 0.769230769         & 7.561929596   & 10              \\ 
\hline
0.00862069       & 0.555555556         & 7.323232323   & 10              \\ 
\hline
0.00862069       & 0.555555556         & 7.323232323   & 10              \\ 
\hline
0.009482759      & 0.55                & 7.25          & 11              \\ 
\hline
0.009482759      & 0.733333333         & 7.209039548   & 11              \\ 
\hline
0.012068966      & 0.7                 & 6.881355932   & 14              \\ 
\hline
0.00862069       & 0.666666667         & 6.553672316   & 10              \\ 
\hline
0.00862069       & 0.555555556         & 5.461393597   & 10              \\
\hline
\end{tabular}
\end{table}

\begin{table}
\centering
\caption{Οι δέκα (10) ισχυρότεροι κανόνες για επίδοση \textit{άριστα}}
\label{table_best}
\begin{tabular}{|l|l|l|l|} 
\hline
\textbf{$support$} & \textbf{$confidence$}  & \textbf{$lift$} & \textbf{$count$}  \\ 
\hline
0.00862069       & 0.714285714 & 18.83116883   & 10              \\ 
\hline
0.00862069       & 0.666666667 & 17.57575758   & 10              \\ 
\hline
0.00862069       & 0.666666667 & 17.57575758   & 10              \\ 
\hline
0.00862069       & 0.666666667 & 17.57575758   & 10              \\ 
\hline
0.00862069       & 0.588235294 & 17.05882353   & 10              \\ 
\hline
0.00862069       & 0.666666667 & 16.45390071   & 10              \\ 
\hline
0.00862069       & 0.666666667 & 16.45390071   & 10              \\ 
\hline
0.011206897      & 0.619047619 & 16.32034632   & 13              \\ 
\hline
0.011206897      & 0.619047619 & 16.32034632   & 13              \\
\hline
\end{tabular}
\end{table}

\clearpage
\section{Επίλογος}

\paragraph{}

Στην εργασία αυτή, στο κεφάλαιο 3 προσπαθήσαμε να κατανοήσουμε καλύτερα την βαθμολογική συμπεριφορά των φοιτητών αλλά και να προσεγγίσουμε ποσοτικά το βαθμό δυσκολίας των μαθημάτων. Στη συνέχεια στο κεφάλαιο 4 με την εφαρμογή του  αλγόριθμου $Apriori$ και την εξόρυξη κανόνων συσχέτισης προσπαθήσαμε να κατανοήσουμε σε πια μαθήματα οι φοιτητές τείνουν να έχουν υψηλές επιδόσεις. 
Χρησιμοποιήθηκε μόνο η μέθοδος της εξόρυξης κανόνων συσχέτισης για την εξόρυξη χρήσιμης πληροφορίας, ωστόσο πρέπει να ληφθεί υπόψιν ότι θα μπορούσαν να χρησιμοποιηθούν και άλλες μέθοδοι εξόρυξης δεδομένων για μία πιο ολοκληρωμένη αξιοποίηση των δεδομένων.

\paragraph{}
Πέρα από την εξόρυξη γνώσης από τα ακαδημαϊκά δεδομένα, είναι πολύ σημαντικό αυτή η γνώση να εφαρμόζεται άμεσα στην εκπαιδευτική διαδικασία. Συνεπώς μελλοντικές  εργασίες πρέπει να έχουν ως αντικείμενό τους τη δημιουργία νέων τρόπων για την αποτελεσματική εφαρμογή της, ήδη εξορυγμένης γνώσης, στα εκπαιδευτικά ιδρύματα.\\
Επόμενο βήμα αυτής της εργασίας θα ήταν αυτή η καινούργια πληροφορία να χρησιμοποιηθεί ως βάση για τη δημιουργία μιας  πλατφόρμας, της οποίας στόχος θα είναι να προτείνει πιθανές κατευθύνσεις στους φοιτητές με βάση τα μαθήματα που είχαν υψηλές  επιδόσεις.  
Η ανάλυση δεδομένων σε εκπαιδευτικά δεδομένα είναι ένας τομέας γεμάτος προκλήσεις αλλά ταυτόχρονα και ζωτικής σημασίας, για την βελτίωση  της εκπαιδευτικής διαδικασίας των πανεπιστημιακών ιδρυμάτων τα επόμενα χρόνια. 

\clearpage

\begin{landscape}
\section*{Παράρτημα Α} 
\addcontentsline{toc}{section}{Παράρτημα Α}
% Long Table

\begin{center}
\begin{longtable}{|l|l|}
%======================================
\caption{Τίτλοι Μαθημάτων των Π3 και Π4 προγραμμάτων σπουδών με τα αντίστοιχά τους στο Π5.} \label{tab:long} \\

\hline \multicolumn{1}{|c|}{\textbf{Τίτλος Μαθήματος (Π3,Π4)}} & \multicolumn{1}{c|}{\textbf{Τίτλος Μαθήματος (Π5)}} \\ \hline 
\endfirsthead

\multicolumn{1}{c}
{{\bfseries \tablename\ \thetable{} -- Συνέχεια από προηγούμενη σελίδα}} \\
\hline \multicolumn{1}{|c|}{\textbf{Τίτλος Μαθήματος (Π3,Π4)}} & \multicolumn{1}{c|}{\textbf{Τίτλος Μαθήματος (Π5)}} \\ \hline 
\endhead


\hline \multicolumn{1}{|r|}{{Συνέχεια στην επόμενη σελίδα}} \\ \hline
\endfoot

\hline \hline
\endlastfoot

\hline
ΑΓΓΛΙΚΗ ΟΡΟΛΟΓΙΑ                                                                                                &                                                                                                                      \\ 
\hline
Αλγοριθμική \& Προγραμματισμός - Ε                                                                              & Αλγοριθμική και Προγραμματισμός - Ε                                                                                  \\ 
\hline
Αλγοριθμική \& Προγραμματισμός - Θ                                                                              & Αλγοριθμική και Προγραμματισμός - Θ                                                                                  \\ 
\hline
\begin{tabular}[c]{@{}l@{}}Αλληλεπίδραση Αθρώπου-Μηχανής \& Ανάπτυξη \\Διεπιφανειών Χρήστη - Ε\end{tabular}     & \begin{tabular}[c]{@{}l@{}}Αλληλεπίδραση Ανθρώπου-Μηχανής και Ανάπτυξη\\ ΔιεπιφανεώνΧρήστη - Ε\end{tabular}          \\ 
\hline
\begin{tabular}[c]{@{}l@{}}Αλληλεπίδραση Αθρώπου-Μηχανής \& Ανάπτυξη\\ Διεπιφανειών Χρήστη - Θ\end{tabular}     & \begin{tabular}[c]{@{}l@{}}Αλληλεπίδραση Ανθρώπου-Μηχανής και Ανάπτυξη\\ ΔιεπιφανεώνΧρήστη - Θ\end{tabular}          \\ 
\hline
\begin{tabular}[c]{@{}l@{}}Ανάπτυξη Διαδικτυακών Συστημάτων \&\\ Εφαργογών - Ε\end{tabular}                     & Ανάπτυξη Διαδικτυακών Συστ. και Εφαρμογών - Ε                                                                        \\ 
\hline
\begin{tabular}[c]{@{}l@{}}Ανάπτυξη Διαδικτυακών Συστημάτων \&\\ Εφαργογών - Θ\end{tabular}                     & Ανάπτυξη Διαδικτυακών Συστ. και Εφαρμογών - Θ                                                                        \\ 
\hline
ΑΝΑΠΤΥΞΗ ΔΙΕΠΙΦΑΝΕΙΩΝ ΧΡΗΣΤΗ-Ε                                                                                  & \begin{tabular}[c]{@{}l@{}}Αλληλεπίδραση Ανθρώπου-Μηχανής και Ανάπτυξη \\ΔιεπιφανεώνΧρήστη - Ε\end{tabular}          \\ 
\hline
ΑΝΑΠΤΥΞΗ ΔΙΕΠΙΦΑΝΕΙΩΝ ΧΡΗΣΤΗ-Θ                                                                                  & \begin{tabular}[c]{@{}l@{}}Αλληλεπίδραση Ανθρώπου-Μηχανής και Ανάπτυξη \\ΔιεπιφανεώνΧρήστη - Θ\end{tabular}          \\ 
\hline
ΑΝΑΠΤΥΞΗ ΚΑΙ ΔΙΑΧΕΙΡΙΣΗ ΕΦΑΡΜΟΓΩΝ-Ε                                                                             & \begin{tabular}[c]{@{}l@{}}Ανάπτυξη και Διαχείριση Ολοκληρωμένων Πληροφ. \\Συστημάτων καιΕφαρμογών - Ε\end{tabular}  \\ 
\hline
ΑΝΑΠΤΥΞΗ ΚΑΙ ΔΙΑΧΕΙΡΙΣΗ ΕΦΑΡΜΟΓΩΝ-Θ                                                                             & \begin{tabular}[c]{@{}l@{}}Ανάπτυξη και Διαχείριση Ολοκληρωμένων Πληροφ.\\ Συστημάτων καιΕφαρμογών - Θ\end{tabular}  \\ 
\hline
\begin{tabular}[c]{@{}l@{}}Ανάπτυξη και Διαχείριση Ολοκληρωμένων Πλ.\\ Συστημάτων \& Εφαρμογών - Ε\end{tabular} & \begin{tabular}[c]{@{}l@{}}Ανάπτυξη και Διαχείριση Ολοκληρωμένων Πληροφ.\\ Συστημάτων καιΕφαρμογών - Ε\end{tabular}  \\ 
\hline
\begin{tabular}[c]{@{}l@{}}Ανάπτυξη και Διαχείριση Ολοκληρωμένων Πλ.\\ Συστημάτων \& Εφαρμογών - Θ\end{tabular} & \begin{tabular}[c]{@{}l@{}}Ανάπτυξη και Διαχείριση Ολοκληρωμένων Πληροφ. \\Συστημάτων καιΕφαρμογών - Θ\end{tabular}  \\ 
\hline
Αντικειμενοστρεφής Προγραμματισμός - Ε                                                                          & Αντικειμενοστραφής Προγραμματισμός - Ε                                                                               \\ 
\hline
Αντικειμενοστρεφής Προγραμματισμός - Θ                                                                          & Αντικειμενοστραφής Προγραμματισμός - Θ                                                                               \\ 
\hline
\begin{tabular}[c]{@{}l@{}}Αριθ. Ανάλυση και Προγ/μος Επιστημονικών\\ Εφαρμογών - Ε\end{tabular}                & Αριθ. Ανάλυση και Προγρ/μός Επιστ. Εφαρμογών - Ε                                                                     \\ 
\hline
\begin{tabular}[c]{@{}l@{}}Αριθ. Ανάλυση και Προγ/μος Επιστημονικών\\ Εφαρμογών - Θ\end{tabular}                & Αριθ. Ανάλυση και Προγρ/μός Επιστ. Εφαρμογών - Θ                                                                     \\ 
\hline
ΑΡΙΘΜ ΑΝΑΛΥΣΗ \& ΠΡΟΓΡ. ΕΠΙΣΤΗΜΟΝ. ΕΦ.-Ε                                                                        & Αριθ. Ανάλυση και Προγρ/μός Επιστ. Εφαρμογών - Ε                                                                     \\ 
\hline
ΑΡΙΘΜ ΑΝΑΛΥΣΗ \& ΠΡΟΓΡ. ΕΠΙΣΤΗΜΟΝ. ΕΦ.-Θ                                                                        & Αριθ. Ανάλυση και Προγρ/μός Επιστ. Εφαρμογών - Θ                                                                     \\ 
\hline
Ασύρματα και Κινητά Δίκτυα Επικοινωνιών                                                                         & Δίκτυα Ασύρματων και Κινητών Επικοινωνιών                                                                            \\ 
\hline
ΑΣΦΑΛΕΙΑ ΠΛΗΡΟΦΟΡΙΑΚΩΝ ΣΥΣΤΗΜΑΤΩΝ-Ε                                                                             & Ασφάλεια Πληροφοριακών Συστημάτων - Ε                                                                                \\ 
\hline
ΑΣΦΑΛΕΙΑ ΠΛΗΡΟΦΟΡΙΑΚΩΝ ΣΥΣΤΗΜΑΤΩΝ-Θ                                                                             & Ασφάλεια Πληροφοριακών Συστημάτων - Θ                                                                                \\ 
\hline
Ασφάλεια Πληροφορικών Συστημάτων - Ε                                                                            & Ασφάλεια Πληροφοριακών Συστημάτων - Ε                                                                                \\ 
\hline
Ασφάλεια Πληροφορικών Συστημάτων - Θ                                                                            & Ασφάλεια Πληροφοριακών Συστημάτων - Θ                                                                                \\ 
\hline
ΒΑΣΕΙΣ ΔΕΔΟΜΕΝΩΝ Ι-Ε                                                                                            & Συστήματα Διαχείρισης Βάσεων Δεδομένων - Ε                                                                           \\ 
\hline
ΒΑΣΕΙΣ ΔΕΔΟΜΕΝΩΝ Ι-Θ                                                                                            & Συστήματα Διαχείρισης Βάσεων Δεδομένων - Θ                                                                           \\ 
\hline
ΒΑΣΕΙΣ ΔΕΔΟΜΕΝΩΝ ΙΙ-Ε                                                                                           & Τεχνολογία Βάσεων Δεδομένων - Ε                                                                                      \\ 
\hline
ΒΑΣΕΙΣ ΔΕΔΟΜΕΝΩΝ ΙΙ-Θ                                                                                           & Τεχνολογία Βάσεων Δεδομένων - Θ                                                                                      \\ 
\hline
ΓΛΩΣΣΕΣ ΚΑΙ ΜΕΤΑΓΛΩΤΤΙΣΤΕΣ-Ε                                                                                    & \#N/A                                                                                                                \\ 
\hline
ΓΛΩΣΣΕΣ ΚΑΙ ΜΕΤΑΓΛΩΤΤΙΣΤΕΣ-Θ                                                                                    & \#N/A                                                                                                                \\ 
\hline
Γλώσσες και Τεχνολογίες Ιστού - Ε                                                                               & Γλώσσες και Τεχνολογίες Ιστού - Ε                                                                                    \\ 
\hline
Γλώσσες και Τεχνολογίες Ιστού - Θ                                                                               & Γλώσσες και Τεχνολογίες Ιστού - Θ                                                                                    \\ 
\hline
Γραφικά Υπολογιστών                                                                                             & Γραφικά Υπολογιστών                                                                                                  \\ 
\hline
ΓΡΑΦΙΚΑ ΥΠΟΛΟΓΙΣΤΩΝ-Ε                                                                                           & Γραφικά Υπολογιστών                                                                                                  \\ 
\hline
ΓΡΑΦΙΚΑ ΥΠΟΛΟΓΙΣΤΩΝ-Θ                                                                                           & Γραφικά Υπολογιστών                                                                                                  \\ 
\hline
Δεξιότητες Επικοινωνίας/Κοινωνικά Δίκτυα - Ε                                                                    & Δεξιότητες Επικοινωνίας/Κοινωνικά Δίκτυα - Ε                                                                         \\ 
\hline
Δεξιότητες Επικοινωνίας/Κοινωνικά Δίκτυα - Θ                                                                    & Δεξιότητες Επικοινωνίας/Κοινωνικά Δίκτυα - Θ                                                                         \\ 
\hline
ΔΕΞΙΟΤΗΤΕΣ ΕΠΙΚΟΙΝΩΝΙΑΣ-Ε                                                                                       & Δεξιότητες Επικοινωνίας/Κοινωνικά Δίκτυα - Ε                                                                         \\ 
\hline
ΔΕΞΙΟΤΗΤΕΣ ΕΠΙΚΟΙΝΩΝΙΑΣ-Θ                                                                                       & Δεξιότητες Επικοινωνίας/Κοινωνικά Δίκτυα - Θ                                                                         \\ 
\hline
\begin{tabular}[c]{@{}l@{}}Διαδικτυακές Υπηρεσίες Προστιθέμενης Αξίας\\ (e-com/gov/e-learn)\end{tabular}        & Διαδικτυακές Υπηρεσίες Προστιθέμενης Αξίας                                                                           \\ 
\hline
ΔΙΑΚΡΙΤΑ ΜΑΘΗΜΑΤΙΚΑ                                                                                             & Διακριτά Μαθηματικά                                                                                                  \\ 
\hline
Διακριτά Μαθηματικά                                                                                             & Διακριτά Μαθηματικά                                                                                                  \\ 
\hline
ΔΙΔΑΚΤΙΚΕΣ ΜΕΘΟΔΟΙ \& ΔΕΟΝΤΟΛΟΓΙΑ ΕΠΑΓΓΕΛ                                                                       & \#N/A                                                                                                                \\ 
\hline
Δίκτυα Η/Υ - Ε                                                                                                  & Δίκτυα Η/Υ - Ε                                                                                                       \\ 
\hline
Δίκτυα Η/Υ - Θ                                                                                                  & Δίκτυα Η/Υ - Θ                                                                                                       \\ 
\hline
ΔΙΚΤΥΑ Η/Υ-Ε                                                                                                    & Δίκτυα Η/Υ - Ε                                                                                                       \\ 
\hline
ΔΙΚΤΥΑ Η/Υ-Θ                                                                                                    & Δίκτυα Η/Υ - Θ                                                                                                       \\ 
\hline
Δομές Δεδομένων και Ανάλυση Αλγορίθμων - Ε                                                                      & Δομές Δεδομένων και Ανάλυση Αλγορίθμων - Ε                                                                           \\ 
\hline
Δομές Δεδομένων και Ανάλυση Αλγορίθμων - Θ                                                                      & Δομές Δεδομένων και Ανάλυση Αλγορίθμων - Θ                                                                           \\ 
\hline
ΔΟΜΕΣ ΔΕΔΟΜΕΝΩΝ-Ε                                                                                               & Δομές Δεδομένων και Ανάλυση Αλγορίθμων - Ε                                                                           \\ 
\hline
ΔΟΜΕΣ ΔΕΔΟΜΕΝΩΝ-Θ                                                                                               & Δομές Δεδομένων και Ανάλυση Αλγορίθμων - Θ                                                                           \\ 
\hline
Ειδικά θέματα δικτύων I - Ε                                                                                     & Ειδικά Θέματα Δικτύων Ι - Ε                                                                                          \\ 
\hline
Ειδικά θέματα δικτύων I - Θ                                                                                     & Ειδικά Θέματα Δικτύων Ι - Θ                                                                                          \\ 
\hline
Ειδικά θέματα δικτύων II - Ε                                                                                    & Ειδικά Θέματα Δικτύων ΙΙ - Ε                                                                                         \\ 
\hline
Ειδικά θέματα δικτύων II - Θ                                                                                    & Ειδικά Θέματα Δικτύων ΙΙ - Θ                                                                                         \\ 
\hline
ΕΙΔΙΚΑ ΘΕΜΑΤΑ ΔΙΚΤΥΩΝ Ι-Ε                                                                                       & Ειδικά Θέματα Δικτύων Ι - Ε                                                                                          \\ 
\hline
ΕΙΔΙΚΑ ΘΕΜΑΤΑ ΔΙΚΤΥΩΝ Ι-Θ                                                                                       & Ειδικά Θέματα Δικτύων ΙΙ - Θ                                                                                         \\ 
\hline
ΕΙΔΙΚΑ ΘΕΜΑΤΑ ΔΙΚΤΥΩΝ Ι-Θ                                                                                       & Ειδικά Θέματα Δικτύων Ι - Θ                                                                                          \\ 
\hline
ΕΙΔΙΚΑ ΘΕΜΑΤΑ ΔΙΚΤΥΩΝ ΙΙ-Ε                                                                                      & Ειδικά Θέματα Δικτύων ΙΙ - Ε                                                                                         \\ 
\hline
ΕΙΔΙΚΑ ΘΕΜΑΤΑ ΔΙΚΤΥΩΝ ΙΙ-Θ                                                                                      & Ειδικά Θέματα Δικτύων ΙΙ - Θ                                                                                         \\ 
\hline
ΕΙΔΙΚΑ ΘΕΜΑΤΑ ΠΛΗΡΟΦΟΡΙΑΚΩΝ ΣΥΣ.-Ε                                                                              & \#N/A                                                                                                                \\ 
\hline
ΕΙΔΙΚΑ ΘΕΜΑΤΑ ΠΛΗΡΟΦΟΡΙΑΚΩΝ ΣΥΣ.-Θ                                                                              & \#N/A                                                                                                                \\ 
\hline
ΕΙΔΙΚΑ ΘΕΜΑΤΑ ΤΕΧΝΗΤΗΣ ΝΟΗΜ/ΝΗΣ-Ε                                                                               & Τεχνητή Νοημοσύνη: Γλώσσες και Τεχνικές - Ε                                                                          \\ 
\hline
ΕΙΔΙΚΑ ΘΕΜΑΤΑ ΤΕΧΝΗΤΗΣ ΝΟΗΜ/ΝΗΣ-Θ                                                                               & Τεχνητή Νοημοσύνη: Γλώσσες και Τεχνικές - Θ                                                                          \\ 
\hline
Εισαγωγή στα Λειτουργικά Συστήματα - Ε                                                                          & Εισαγωγή στα Λειτουργικά Συστηματα - Ε                                                                               \\ 
\hline
Εισαγωγή στα Λειτουργικά Συστήματα - Θ                                                                          & Εισαγωγή στα Λειτουργικά Συστηματα - Θ                                                                               \\ 
\hline
Εισαγωγή στην Πληροφορική - Ε                                                                                   & Εισαγωγή στην Πληροφορική - Ε                                                                                        \\ 
\hline
Εισαγωγή στην Πληροφορική - Θ                                                                                   & Εισαγωγή στην Πληροφορική - Θ                                                                                        \\ 
\hline
ΕΙΣΑΓΩΓΗ ΣΤΗΝ ΠΛΗΡΟΦΟΡΙΚΗ-Ε                                                                                     & Εισαγωγή στην Πληροφορική - Ε                                                                                        \\ 
\hline
ΕΙΣΑΓΩΓΗ ΣΤΗΝ ΠΛΗΡΟΦΟΡΙΚΗ-Θ                                                                                     & Εισαγωγή στην Πληροφορική - Θ                                                                                        \\ 
\hline
ΕΠΕΞΕΡΓΑΣΙΑ ΣΗΜΑΤΟΣ \& ΕΙΚΟΝΑΣ-Ε                                                                                & \#N/A                                                                                                                \\ 
\hline
ΕΠΕΞΕΡΓΑΣΙΑ ΣΗΜΑΤΟΣ \& ΕΙΚΟΝΑΣ-Θ                                                                                & \#N/A                                                                                                                \\ 
\hline
ΕΠΙΧΕΙΡΗΜΑΤΙΚΟΤΗΤΑ Ι                                                                                            & \#N/A                                                                                                                \\ 
\hline
ΕΠΙΧΕΙΡΗΜΑΤΙΚΟΤΗΤΑ ΙΙ                                                                                           & \#N/A                                                                                                                \\ 
\hline
ΕΠΙΧΕΙΡΗΣΙΑΚΗ ΕΡΕΥΝΑ                                                                                            & Επιχειρησιακή Έρευνα                                                                                                 \\ 
\hline
Επιχειρησιακή Έρευνα                                                                                            & Επιχειρησιακή Έρευνα                                                                                                 \\ 
\hline
Ευφυή Συστήματα                                                                                                 & Ευφυή Συστήματα                                                                                                      \\ 
\hline
ΕΥΦΥΗ ΣΥΣΤΗΜΑΤΑ-Ε                                                                                               & Ευφυή Συστήματα                                                                                                      \\ 
\hline
ΕΥΦΥΗ ΣΥΣΤΗΜΑΤΑ-Θ                                                                                               & Ευφυή Συστήματα                                                                                                      \\ 
\hline
ΗΛΕΚΤΡΟΝΙΚΗ ΜΑΘΗΣΗ-Ε                                                                                            & \#N/A                                                                                                                \\ 
\hline
ΗΛΕΚΤΡΟΝΙΚΗ ΜΑΘΗΣΗ-Θ                                                                                            & \#N/A                                                                                                                \\ 
\hline
ΗΛΕΚΤΡΟΝΙΚΟ ΕΜΠΟΡΙΟ-Ε                                                                                           & Διαδικτυακές Υπηρεσίες Προστιθέμενης Αξίας                                                                           \\ 
\hline
ΗΛΕΚΤΡΟΝΙΚΟ ΕΜΠΟΡΙΟ-Θ                                                                                           & Διαδικτυακές Υπηρεσίες Προστιθέμενης Αξίας                                                                           \\ 
\hline
Θεωρία Λειτουργικών Συστημάτων                                                                                  & Αρχές Σχεδίασης Λειτουργικών Συστημάτων                                                                              \\ 
\hline
ΘΕΩΡΙΑ ΠΙΘΑΝ/ΤΩΝ ΚΑΙ ΣΤΑΤΙΣΤΙΚΗ-Ε                                                                               & Θεωρία Πιθανοτήτων και Στατιστική - Ε                                                                                \\ 
\hline
ΘΕΩΡΙΑ ΠΙΘΑΝ/ΤΩΝ ΚΑΙ ΣΤΑΤΙΣΤΙΚΗ-Θ                                                                               & Θεωρία Πιθανοτήτων και Στατιστική - Θ                                                                                \\ 
\hline
Θεωρία Πιθανοτήτων και Στατιστική - Ε                                                                           & Θεωρία Πιθανοτήτων και Στατιστική - Ε                                                                                \\ 
\hline
Θεωρία Πιθανοτήτων και Στατιστική - Θ                                                                           & Θεωρία Πιθανοτήτων και Στατιστική - Θ                                                                                \\ 
\hline
ΛΕΙΤΟΥΡΓΙΚΑ ΣΥΣΤΗΜΑΤΑ Ι                                                                                         & Αρχές Σχεδίασης Λειτουργικών Συστημάτων                                                                              \\ 
\hline
ΛΕΙΤΟΥΡΓΙΚΑ ΣΥΣΤΗΜΑΤΑ ΙΙ-Ε                                                                                      & Εισαγωγή στα Λειτουργικά Συστηματα - Ε                                                                               \\ 
\hline
ΛΕΙΤΟΥΡΓΙΚΑ ΣΥΣΤΗΜΑΤΑ ΙΙ-Θ                                                                                      & Εισαγωγή στα Λειτουργικά Συστηματα - Θ                                                                               \\ 
\hline
ΜΑΘΗΜΑΤΙΚΗ ΑΝΑΛΥΣΗ                                                                                              & Μαθηματική Ανάλυση και Γραμμική Άλγεβρα                                                                              \\ 
\hline
Μαθηματική Ανάλυση                                                                                              & Μαθηματική Ανάλυση και Γραμμική Άλγεβρα                                                                              \\ 
\hline
ΜΕΘΟΔΟΛΟΓΙΑ ΠΡΟΓΡΑΜΜΑΤΙΣΜΟΥ Ι-Ε                                                                                 & Μεθοδολογίες Προγραμματισμού - Ε                                                                                     \\ 
\hline
ΜΕΘΟΔΟΛΟΓΙΑ ΠΡΟΓΡΑΜΜΑΤΙΣΜΟΥ Ι-Θ                                                                                 & Μεθοδολογίες Προγραμματισμού - Θ                                                                                     \\ 
\hline
Μεθοδολογίες Προγραμματισμού - Ε                                                                                & Μεθοδολογίες Προγραμματισμού - Ε                                                                                     \\ 
\hline
Μεθοδολογίες Προγραμματισμού - Θ                                                                                & Μεθοδολογίες Προγραμματισμού - Θ                                                                                     \\ 
\hline
ΜΕΘΟΔΟΛΟΓΙΕΣ ΠΡΟΓΡΑΜΜΑΤΙΣΜΟΥ ΙΙ-Ε                                                                               & Μεθοδολογίες Προγραμματισμού - Ε                                                                                     \\ 
\hline
ΜΕΘΟΔΟΛΟΓΙΕΣ ΠΡΟΓΡΑΜΜΑΤΙΣΜΟΥ ΙΙ-Θ                                                                               & Μεθοδολογίες Προγραμματισμού - Θ                                                                                     \\ 
\hline
ΜΕΝΤΟΡΕΣ                                                                                                        & \#N/A                                                                                                                \\ 
\hline
ΜΗΧΑΝΙΚΗ ΛΟΓΙΣΜΙΚΟΥ                                                                                             & Μηχανική Λογισμικού Ι - Θ                                                                                            \\ 
\hline
Μηχανική Λογισμικού IΙ                                                                                          & Μηχανική Λογισμικού ΙΙ                                                                                               \\ 
\hline
Μηχανική Λογισμικού Ι - Ε                                                                                       & Μηχανική Λογισμικού Ι - Ε                                                                                            \\ 
\hline
Μηχανική Λογισμικού Ι - Θ                                                                                       & Μηχανική Λογισμικού Ι - Θ                                                                                            \\ 
\hline
Μηχανική Μάθηση - Ε                                                                                             & Μηχανική Μάθηση - Ε                                                                                                  \\ 
\hline
Μηχανική Μάθηση - Θ                                                                                             & Μηχανική Μάθηση - Θ                                                                                                  \\ 
\hline
ΝΕΥΡΩΝΙΚΑ (ΝΕΥΡΟΜΟΡΦΙΚΑ) ΔΙΚΤΥΑ-Ε                                                                               & Μηχανική Μάθηση - Ε                                                                                                  \\ 
\hline
ΝΕΥΡΩΝΙΚΑ (ΝΕΥΡΟΜΟΡΦΙΚΑ) ΔΙΚΤΥΑ-Θ                                                                               & Μηχανική Μάθηση - Θ                                                                                                  \\ 
\hline
ΞΕΝΗ ΓΛΩΣΣΑ Ι                                                                                                   & \#N/A                                                                                                                \\ 
\hline
ΟΙΚΟΝΟΜΙΑ ΤΩΝ ΕΠΙΧ/ΩΝ \& ΟΡΓ/ΣΗ Δ/ΣΗ ΕΠΙΧ                                                                       & \#N/A                                                                                                                \\ 
\hline
ΟΡΓ/ΣΗ \& ΑΡΧ/ΚΗ ΥΠΟΛ/ΚΩΝ ΣΥΣ/ΤΩΝ-Ε                                                                             & Οργάνωση και Αρχιτεκτονική Υπολ. Συστημάτων - Ε                                                                      \\ 
\hline
ΟΡΓ/ΣΗ \& ΑΡΧ/ΚΗ ΥΠΟΛ/ΚΩΝ ΣΥΣ/ΤΩΝ-Θ                                                                             & Οργάνωση και Αρχιτεκτονική Υπολ. Συστημάτων - Θ                                                                      \\ 
\hline
Οργάνωση \& Αρχιτεκτονική Υπολ. Συστημάτων - Ε                                                                  & Οργάνωση και Αρχιτεκτονική Υπολ. Συστημάτων - Ε                                                                      \\ 
\hline
Οργάνωση \& Αρχιτεκτονική Υπολ. Συστημάτων - Θ                                                                  & Οργάνωση και Αρχιτεκτονική Υπολ. Συστημάτων - Θ                                                                      \\ 
\hline
Οργάνωση Δεδομένων \& Εξόρυξη Πληροφορίας                                                                       & Οργάνωση Δεδομένων και Εξόρυξη Πληροφορίας                                                                           \\ 
\hline
ΠΑΡΑΛΛΗΛΑ ΚΑΙ ΚΑΤΑΝΕΜΗΜΕΝΑ ΣΥΣ/ΤΑ-Ε                                                                             & \begin{tabular}[c]{@{}l@{}}Προηγμένες Αρχιτεκτονικές Υπολογιστών και\\ ΠαράλληλαΣυστήματα\end{tabular}               \\ 
\hline
ΠΑΡΑΛΛΗΛΑ ΚΑΙ ΚΑΤΑΝΕΜΗΜΕΝΑ ΣΥΣ/ΤΑ-Θ                                                                             & \begin{tabular}[c]{@{}l@{}}Προηγμένες Αρχιτεκτονικές Υπολογιστών και \\Παράλληλα Συστήματα\end{tabular}              \\ 
\hline
ΠΛΗΡ/ΚΗ ΚΑΙ Κ/ΝΙΑ \&ΤΕΧ/ΚΕΣ ΠΡ/ΣΗΣ ΑΓΟΡΑΣ                                                                       & \#N/A                                                                                                                \\ 
\hline
ΠΛΗΡΟΦΟΡΙΑΚΑ ΣΥΣΤΗΜΑΤΑ ΔΙΟΙΚΗΣΗΣ-Ε                                                                              & Διοίκηση και Διαχείριση Έργων Πληροφορικής                                                                           \\ 
\hline
ΠΛΗΡΟΦΟΡΙΑΚΑ ΣΥΣΤΗΜΑΤΑ ΔΙΟΙΚΗΣΗΣ-Θ                                                                              & Διοίκηση και Διαχείριση Έργων Πληροφορικής                                                                           \\ 
\hline
ΠΛΗΡΟΦΟΡΙΑΚΑ ΣΥΣΤΗΜΑΤΑ Ι                                                                                        & Πληροφοριακά Συστήματα Ι                                                                                             \\ 
\hline
Πληροφοριακά Συστήματα Ι                                                                                        & Πληροφοριακά Συστήματα Ι                                                                                             \\ 
\hline
ΠΛΗΡΟΦΟΡΙΑΚΑ ΣΥΣΤΗΜΑΤΑ ΙΙ                                                                                       & Πληροφοριακά Συστήματα ΙΙ                                                                                            \\ 
\hline
Πληροφοριακά Συστήματα ΙΙ - Θ                                                                                   & Πληροφοριακά Συστήματα ΙΙ                                                                                            \\ 
\hline
ΠΟΙΟΤΗΤΑ \& ΑΞΙΟΠΙΣΤΙΑ ΛΟΓΙΣΜΙΚΟΥ-Ε                                                                             & \#N/A                                                                                                                \\ 
\hline
ΠΟΙΟΤΗΤΑ \& ΑΞΙΟΠΙΣΤΙΑ ΛΟΓΙΣΜΙΚΟΥ-Θ                                                                             & \#N/A                                                                                                                \\ 
\hline
ΠΡΟΓ/ΣΜΟΣ ΔΙΑΔΙΚΤΥΑΚΩΝ ΕΦΑΡ/ΓΩΝ-Ε                                                                               & Ανάπτυξη Διαδικτυακών Συστ. και Εφαρμογών - Ε                                                                        \\ 
\hline
ΠΡΟΓ/ΣΜΟΣ ΔΙΑΔΙΚΤΥΑΚΩΝ ΕΦΑΡ/ΓΩΝ-Θ                                                                               & Ανάπτυξη Διαδικτυακών Συστ. και Εφαρμογών - Θ                                                                        \\ 
\hline
ΠΡΟΓΡΑΜΜΑΤΙΣΜΟΣ ΥΠΟΛΟΓΙΣΤΩΝ Ι-Ε                                                                                 & Αλγοριθμική και Προγραμματισμός - Ε                                                                                  \\ 
\hline
ΠΡΟΓΡΑΜΜΑΤΙΣΜΟΣ ΥΠΟΛΟΓΙΣΤΩΝ Ι-Θ                                                                                 & Αλγοριθμική και Προγραμματισμός - Θ                                                                                  \\ 
\hline
ΠΡΟΓΡΑΜΜΑΤΙΣΜΟΣ ΥΠΟΛΟΓΙΣΤΩΝ ΙΙ-Ε                                                                                & Αντικειμενοστραφής Προγραμματισμός - Ε                                                                               \\ 
\hline
ΠΡΟΓΡΑΜΜΑΤΙΣΜΟΣ ΥΠΟΛΟΓΙΣΤΩΝ ΙΙ-Θ                                                                                & Αντικειμενοστραφής Προγραμματισμός - Θ                                                                               \\ 
\hline
ΠΡΟΗΓΜΕΝΕΣ ΑΡΧΙΤΕΚΤΟΝΙΚΕΣ Η/Υ-Ε                                                                                 & \begin{tabular}[c]{@{}l@{}}Προηγμένες Αρχιτεκτονικές Υπολογιστών και \\Παράλληλα Συστήματα\end{tabular}              \\ 
\hline
ΠΡΟΗΓΜΕΝΕΣ ΑΡΧΙΤΕΚΤΟΝΙΚΕΣ Η/Υ-Θ                                                                                 & \begin{tabular}[c]{@{}l@{}}Προηγμένες Αρχιτεκτονικές Υπολογιστών και \\ΠαράλληλαΣυστήματα\end{tabular}               \\ 
\hline
\begin{tabular}[c]{@{}l@{}}Προηγμένες Αρχιτεκτονικές Υπολογιστών και \\Παράλληλα Συστήματα\end{tabular}         & \begin{tabular}[c]{@{}l@{}}Προηγμένες Αρχιτεκτονικές Υπολογιστών και \\ΠαράλληλαΣυστήματα\end{tabular}               \\ 
\hline
ΣΥΣΤΗΜΑΤΑ ΓΕΩΓΡΑΦΙΚΩΝ ΠΛΗΡΟΦΟΡΙΩΝ-Ε                                                                             & \#N/A                                                                                                                \\ 
\hline
ΣΥΣΤΗΜΑΤΑ ΓΕΩΓΡΑΦΙΚΩΝ ΠΛΗΡΟΦΟΡΙΩΝ-Θ                                                                             & \#N/A                                                                                                                \\ 
\hline
Συστήματα Διαχείρισης Βάσεων Δεδομένων - Ε                                                                      & Συστήματα Διαχείρισης Βάσεων Δεδομένων - Ε                                                                           \\ 
\hline
Συστήματα Διαχείρισης Βάσεων Δεδομένων - Θ                                                                      & Συστήματα Διαχείρισης Βάσεων Δεδομένων - Θ                                                                           \\ 
\hline
Τεχνητή Νοημοσύνη (Γλώσσες \& Τεχνικές) - Ε                                                                     & Τεχνητή Νοημοσύνη: Γλώσσες και Τεχνικές - Ε                                                                          \\ 
\hline
Τεχνητή Νοημοσύνη (Γλώσσες \& Τεχνικές) - Θ                                                                     & Τεχνητή Νοημοσύνη: Γλώσσες και Τεχνικές - Θ                                                                          \\ 
\hline
ΤΕΧΝΗΤΗ ΝΟΗΜΟΣΥΝΗ-Ε                                                                                             & Τεχνητή Νοημοσύνη: Γλώσσες και Τεχνικές - Ε                                                                          \\ 
\hline
ΤΕΧΝΗΤΗ ΝΟΗΜΟΣΥΝΗ-Θ                                                                                             & Τεχνητή Νοημοσύνη: Γλώσσες και Τεχνικές - Θ                                                                          \\ 
\hline
Τεχνολογία Βάσεων Δεδομένων - Ε                                                                                 & Τεχνολογία Βάσεων Δεδομένων - Ε                                                                                      \\ 
\hline
Τεχνολογία Βάσεων Δεδομένων - Θ                                                                                 & Τεχνολογία Βάσεων Δεδομένων - Θ                                                                                      \\ 
\hline
Τεχνολογία Πολυμέσων - Ε                                                                                        & Τεχνολογία Πολυμέσων - Ε                                                                                             \\ 
\hline
Τεχνολογία Πολυμέσων - Θ                                                                                        & Τεχνολογία Πολυμέσων - Θ                                                                                             \\ 
\hline
ΤΕΧΝΟΛΟΓΙΑ ΠΟΛΥΜΕΣΩΝ-Ε                                                                                          & Τεχνολογία Πολυμέσων - Ε                                                                                             \\ 
\hline
ΤΕΧΝΟΛΟΓΙΑ ΠΟΛΥΜΕΣΩΝ-Θ                                                                                          & Τεχνολογία Πολυμέσων - Θ                                                                                             \\ 
\hline
ΤΗΛΕΠΙΚΟΙΝΩΝΙΕΣ \& ΔΙΚΤΥΑ-Ε                                                                                     & Τηλεπικοινωνίες και Δίκτυα Υπολογιστών - Ε                                                                           \\ 
\hline
ΤΗΛΕΠΙΚΟΙΝΩΝΙΕΣ \& ΔΙΚΤΥΑ-Θ                                                                                     & Τηλεπικοινωνίες και Δίκτυα Υπολογιστών - Θ                                                                           \\ 
\hline
ΤΗΛΕΠΙΚΟΙΝΩΝΙΕΣ ΚΑΙ ΔΙΚΤΥΑ Η/Υ                                                                                  & Τηλεπικοινωνίες και Δίκτυα Υπολογιστών - Θ                                                                           \\ 
\hline
Τηλεπικοινωνίες και Δίκτυα Υπολογιστών - Ε                                                                      & Τηλεπικοινωνίες και Δίκτυα Υπολογιστών - Ε                                                                           \\ 
\hline
Τηλεπικοινωνίες και Δίκτυα Υπολογιστών - Θ                                                                      & Τηλεπικοινωνίες και Δίκτυα Υπολογιστών - Θ                                                                           \\ 
\hline
Ψηφιακά Συστήματα                                                                                               & Ψηφιακά Συστήματα                                                                                                    \\ 
\hline
ΨΗΦΙΑΚΑ ΣΥΣΤΗΜΑΤΑ-Ε                                                                                             & Ψηφιακά Συστήματα                                                                                                    \\ 
\hline
ΨΗΦΙΑΚΑ ΣΥΣΤΗΜΑΤΑ-Θ                                                                                             & Ψηφιακά Συστήματα                                                                                                    \\
\hline


%=====================================
\end{longtable}
\end{center}




\end{landscape}
\clearpage

\section*{Αναφορές}
\addcontentsline{toc}{section}{Αναφορές}
\selectlanguage{english}
\printbibliography

\selectlanguage{greek}
\clearpage

\end{document}














